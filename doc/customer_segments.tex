
% Default to the notebook output style

    


% Inherit from the specified cell style.




    
\documentclass{article}

    
    
    \usepackage{graphicx} % Used to insert images
    \usepackage{adjustbox} % Used to constrain images to a maximum size 
    \usepackage{color} % Allow colors to be defined
    \usepackage{enumerate} % Needed for markdown enumerations to work
    \usepackage{geometry} % Used to adjust the document margins
    \usepackage{amsmath} % Equations
    \usepackage{amssymb} % Equations
    \usepackage{eurosym} % defines \euro
    \usepackage[mathletters]{ucs} % Extended unicode (utf-8) support
    \usepackage[utf8x]{inputenc} % Allow utf-8 characters in the tex document
    \usepackage{fancyvrb} % verbatim replacement that allows latex
    \usepackage{grffile} % extends the file name processing of package graphics 
                         % to support a larger range 
    % The hyperref package gives us a pdf with properly built
    % internal navigation ('pdf bookmarks' for the table of contents,
    % internal cross-reference links, web links for URLs, etc.)
    \usepackage{hyperref}
    \usepackage{longtable} % longtable support required by pandoc >1.10
    \usepackage{booktabs}  % table support for pandoc > 1.12.2
    \usepackage{ulem} % ulem is needed to support strikethroughs (\sout)
    \usepackage{textcomp} % Additional dependency for \textquotesingle
    

    
    
    \definecolor{orange}{cmyk}{0,0.4,0.8,0.2}
    \definecolor{darkorange}{rgb}{.71,0.21,0.01}
    \definecolor{darkgreen}{rgb}{.12,.54,.11}
    \definecolor{myteal}{rgb}{.26, .44, .56}
    \definecolor{gray}{gray}{0.45}
    \definecolor{lightgray}{gray}{.95}
    \definecolor{mediumgray}{gray}{.8}
    \definecolor{inputbackground}{rgb}{.95, .95, .85}
    \definecolor{outputbackground}{rgb}{.95, .95, .95}
    \definecolor{traceback}{rgb}{1, .95, .95}
    % ansi colors
    \definecolor{red}{rgb}{.6,0,0}
    \definecolor{green}{rgb}{0,.65,0}
    \definecolor{brown}{rgb}{0.6,0.6,0}
    \definecolor{blue}{rgb}{0,.145,.698}
    \definecolor{purple}{rgb}{.698,.145,.698}
    \definecolor{cyan}{rgb}{0,.698,.698}
    \definecolor{lightgray}{gray}{0.5}
    
    % bright ansi colors
    \definecolor{darkgray}{gray}{0.25}
    \definecolor{lightred}{rgb}{1.0,0.39,0.28}
    \definecolor{lightgreen}{rgb}{0.48,0.99,0.0}
    \definecolor{lightblue}{rgb}{0.53,0.81,0.92}
    \definecolor{lightpurple}{rgb}{0.87,0.63,0.87}
    \definecolor{lightcyan}{rgb}{0.5,1.0,0.83}
    
    % commands and environments needed by pandoc snippets
    % extracted from the output of `pandoc -s`
    \providecommand{\tightlist}{%
      \setlength{\itemsep}{0pt}\setlength{\parskip}{0pt}}
    \DefineVerbatimEnvironment{Highlighting}{Verbatim}{commandchars=\\\{\}}
    % Add ',fontsize=\small' for more characters per line
    \newenvironment{Shaded}{}{}
    \newcommand{\KeywordTok}[1]{\textcolor[rgb]{0.00,0.44,0.13}{\textbf{{#1}}}}
    \newcommand{\DataTypeTok}[1]{\textcolor[rgb]{0.56,0.13,0.00}{{#1}}}
    \newcommand{\DecValTok}[1]{\textcolor[rgb]{0.25,0.63,0.44}{{#1}}}
    \newcommand{\BaseNTok}[1]{\textcolor[rgb]{0.25,0.63,0.44}{{#1}}}
    \newcommand{\FloatTok}[1]{\textcolor[rgb]{0.25,0.63,0.44}{{#1}}}
    \newcommand{\CharTok}[1]{\textcolor[rgb]{0.25,0.44,0.63}{{#1}}}
    \newcommand{\StringTok}[1]{\textcolor[rgb]{0.25,0.44,0.63}{{#1}}}
    \newcommand{\CommentTok}[1]{\textcolor[rgb]{0.38,0.63,0.69}{\textit{{#1}}}}
    \newcommand{\OtherTok}[1]{\textcolor[rgb]{0.00,0.44,0.13}{{#1}}}
    \newcommand{\AlertTok}[1]{\textcolor[rgb]{1.00,0.00,0.00}{\textbf{{#1}}}}
    \newcommand{\FunctionTok}[1]{\textcolor[rgb]{0.02,0.16,0.49}{{#1}}}
    \newcommand{\RegionMarkerTok}[1]{{#1}}
    \newcommand{\ErrorTok}[1]{\textcolor[rgb]{1.00,0.00,0.00}{\textbf{{#1}}}}
    \newcommand{\NormalTok}[1]{{#1}}
    
    % Additional commands for more recent versions of Pandoc
    \newcommand{\ConstantTok}[1]{\textcolor[rgb]{0.53,0.00,0.00}{{#1}}}
    \newcommand{\SpecialCharTok}[1]{\textcolor[rgb]{0.25,0.44,0.63}{{#1}}}
    \newcommand{\VerbatimStringTok}[1]{\textcolor[rgb]{0.25,0.44,0.63}{{#1}}}
    \newcommand{\SpecialStringTok}[1]{\textcolor[rgb]{0.73,0.40,0.53}{{#1}}}
    \newcommand{\ImportTok}[1]{{#1}}
    \newcommand{\DocumentationTok}[1]{\textcolor[rgb]{0.73,0.13,0.13}{\textit{{#1}}}}
    \newcommand{\AnnotationTok}[1]{\textcolor[rgb]{0.38,0.63,0.69}{\textbf{\textit{{#1}}}}}
    \newcommand{\CommentVarTok}[1]{\textcolor[rgb]{0.38,0.63,0.69}{\textbf{\textit{{#1}}}}}
    \newcommand{\VariableTok}[1]{\textcolor[rgb]{0.10,0.09,0.49}{{#1}}}
    \newcommand{\ControlFlowTok}[1]{\textcolor[rgb]{0.00,0.44,0.13}{\textbf{{#1}}}}
    \newcommand{\OperatorTok}[1]{\textcolor[rgb]{0.40,0.40,0.40}{{#1}}}
    \newcommand{\BuiltInTok}[1]{{#1}}
    \newcommand{\ExtensionTok}[1]{{#1}}
    \newcommand{\PreprocessorTok}[1]{\textcolor[rgb]{0.74,0.48,0.00}{{#1}}}
    \newcommand{\AttributeTok}[1]{\textcolor[rgb]{0.49,0.56,0.16}{{#1}}}
    \newcommand{\InformationTok}[1]{\textcolor[rgb]{0.38,0.63,0.69}{\textbf{\textit{{#1}}}}}
    \newcommand{\WarningTok}[1]{\textcolor[rgb]{0.38,0.63,0.69}{\textbf{\textit{{#1}}}}}
    
    
    % Define a nice break command that doesn't care if a line doesn't already
    % exist.
    \def\br{\hspace*{\fill} \\* }
    % Math Jax compatability definitions
    \def\gt{>}
    \def\lt{<}
    % Document parameters
    \title{customer\_segments}
    
    
    

    % Pygments definitions
    
\makeatletter
\def\PY@reset{\let\PY@it=\relax \let\PY@bf=\relax%
    \let\PY@ul=\relax \let\PY@tc=\relax%
    \let\PY@bc=\relax \let\PY@ff=\relax}
\def\PY@tok#1{\csname PY@tok@#1\endcsname}
\def\PY@toks#1+{\ifx\relax#1\empty\else%
    \PY@tok{#1}\expandafter\PY@toks\fi}
\def\PY@do#1{\PY@bc{\PY@tc{\PY@ul{%
    \PY@it{\PY@bf{\PY@ff{#1}}}}}}}
\def\PY#1#2{\PY@reset\PY@toks#1+\relax+\PY@do{#2}}

\expandafter\def\csname PY@tok@gd\endcsname{\def\PY@tc##1{\textcolor[rgb]{0.63,0.00,0.00}{##1}}}
\expandafter\def\csname PY@tok@gu\endcsname{\let\PY@bf=\textbf\def\PY@tc##1{\textcolor[rgb]{0.50,0.00,0.50}{##1}}}
\expandafter\def\csname PY@tok@gt\endcsname{\def\PY@tc##1{\textcolor[rgb]{0.00,0.27,0.87}{##1}}}
\expandafter\def\csname PY@tok@gs\endcsname{\let\PY@bf=\textbf}
\expandafter\def\csname PY@tok@gr\endcsname{\def\PY@tc##1{\textcolor[rgb]{1.00,0.00,0.00}{##1}}}
\expandafter\def\csname PY@tok@cm\endcsname{\let\PY@it=\textit\def\PY@tc##1{\textcolor[rgb]{0.25,0.50,0.50}{##1}}}
\expandafter\def\csname PY@tok@vg\endcsname{\def\PY@tc##1{\textcolor[rgb]{0.10,0.09,0.49}{##1}}}
\expandafter\def\csname PY@tok@vi\endcsname{\def\PY@tc##1{\textcolor[rgb]{0.10,0.09,0.49}{##1}}}
\expandafter\def\csname PY@tok@mh\endcsname{\def\PY@tc##1{\textcolor[rgb]{0.40,0.40,0.40}{##1}}}
\expandafter\def\csname PY@tok@cs\endcsname{\let\PY@it=\textit\def\PY@tc##1{\textcolor[rgb]{0.25,0.50,0.50}{##1}}}
\expandafter\def\csname PY@tok@ge\endcsname{\let\PY@it=\textit}
\expandafter\def\csname PY@tok@vc\endcsname{\def\PY@tc##1{\textcolor[rgb]{0.10,0.09,0.49}{##1}}}
\expandafter\def\csname PY@tok@il\endcsname{\def\PY@tc##1{\textcolor[rgb]{0.40,0.40,0.40}{##1}}}
\expandafter\def\csname PY@tok@go\endcsname{\def\PY@tc##1{\textcolor[rgb]{0.53,0.53,0.53}{##1}}}
\expandafter\def\csname PY@tok@cp\endcsname{\def\PY@tc##1{\textcolor[rgb]{0.74,0.48,0.00}{##1}}}
\expandafter\def\csname PY@tok@gi\endcsname{\def\PY@tc##1{\textcolor[rgb]{0.00,0.63,0.00}{##1}}}
\expandafter\def\csname PY@tok@gh\endcsname{\let\PY@bf=\textbf\def\PY@tc##1{\textcolor[rgb]{0.00,0.00,0.50}{##1}}}
\expandafter\def\csname PY@tok@ni\endcsname{\let\PY@bf=\textbf\def\PY@tc##1{\textcolor[rgb]{0.60,0.60,0.60}{##1}}}
\expandafter\def\csname PY@tok@nl\endcsname{\def\PY@tc##1{\textcolor[rgb]{0.63,0.63,0.00}{##1}}}
\expandafter\def\csname PY@tok@nn\endcsname{\let\PY@bf=\textbf\def\PY@tc##1{\textcolor[rgb]{0.00,0.00,1.00}{##1}}}
\expandafter\def\csname PY@tok@no\endcsname{\def\PY@tc##1{\textcolor[rgb]{0.53,0.00,0.00}{##1}}}
\expandafter\def\csname PY@tok@na\endcsname{\def\PY@tc##1{\textcolor[rgb]{0.49,0.56,0.16}{##1}}}
\expandafter\def\csname PY@tok@nb\endcsname{\def\PY@tc##1{\textcolor[rgb]{0.00,0.50,0.00}{##1}}}
\expandafter\def\csname PY@tok@nc\endcsname{\let\PY@bf=\textbf\def\PY@tc##1{\textcolor[rgb]{0.00,0.00,1.00}{##1}}}
\expandafter\def\csname PY@tok@nd\endcsname{\def\PY@tc##1{\textcolor[rgb]{0.67,0.13,1.00}{##1}}}
\expandafter\def\csname PY@tok@ne\endcsname{\let\PY@bf=\textbf\def\PY@tc##1{\textcolor[rgb]{0.82,0.25,0.23}{##1}}}
\expandafter\def\csname PY@tok@nf\endcsname{\def\PY@tc##1{\textcolor[rgb]{0.00,0.00,1.00}{##1}}}
\expandafter\def\csname PY@tok@si\endcsname{\let\PY@bf=\textbf\def\PY@tc##1{\textcolor[rgb]{0.73,0.40,0.53}{##1}}}
\expandafter\def\csname PY@tok@s2\endcsname{\def\PY@tc##1{\textcolor[rgb]{0.73,0.13,0.13}{##1}}}
\expandafter\def\csname PY@tok@nt\endcsname{\let\PY@bf=\textbf\def\PY@tc##1{\textcolor[rgb]{0.00,0.50,0.00}{##1}}}
\expandafter\def\csname PY@tok@nv\endcsname{\def\PY@tc##1{\textcolor[rgb]{0.10,0.09,0.49}{##1}}}
\expandafter\def\csname PY@tok@s1\endcsname{\def\PY@tc##1{\textcolor[rgb]{0.73,0.13,0.13}{##1}}}
\expandafter\def\csname PY@tok@ch\endcsname{\let\PY@it=\textit\def\PY@tc##1{\textcolor[rgb]{0.25,0.50,0.50}{##1}}}
\expandafter\def\csname PY@tok@m\endcsname{\def\PY@tc##1{\textcolor[rgb]{0.40,0.40,0.40}{##1}}}
\expandafter\def\csname PY@tok@gp\endcsname{\let\PY@bf=\textbf\def\PY@tc##1{\textcolor[rgb]{0.00,0.00,0.50}{##1}}}
\expandafter\def\csname PY@tok@sh\endcsname{\def\PY@tc##1{\textcolor[rgb]{0.73,0.13,0.13}{##1}}}
\expandafter\def\csname PY@tok@ow\endcsname{\let\PY@bf=\textbf\def\PY@tc##1{\textcolor[rgb]{0.67,0.13,1.00}{##1}}}
\expandafter\def\csname PY@tok@sx\endcsname{\def\PY@tc##1{\textcolor[rgb]{0.00,0.50,0.00}{##1}}}
\expandafter\def\csname PY@tok@bp\endcsname{\def\PY@tc##1{\textcolor[rgb]{0.00,0.50,0.00}{##1}}}
\expandafter\def\csname PY@tok@c1\endcsname{\let\PY@it=\textit\def\PY@tc##1{\textcolor[rgb]{0.25,0.50,0.50}{##1}}}
\expandafter\def\csname PY@tok@o\endcsname{\def\PY@tc##1{\textcolor[rgb]{0.40,0.40,0.40}{##1}}}
\expandafter\def\csname PY@tok@kc\endcsname{\let\PY@bf=\textbf\def\PY@tc##1{\textcolor[rgb]{0.00,0.50,0.00}{##1}}}
\expandafter\def\csname PY@tok@c\endcsname{\let\PY@it=\textit\def\PY@tc##1{\textcolor[rgb]{0.25,0.50,0.50}{##1}}}
\expandafter\def\csname PY@tok@mf\endcsname{\def\PY@tc##1{\textcolor[rgb]{0.40,0.40,0.40}{##1}}}
\expandafter\def\csname PY@tok@err\endcsname{\def\PY@bc##1{\setlength{\fboxsep}{0pt}\fcolorbox[rgb]{1.00,0.00,0.00}{1,1,1}{\strut ##1}}}
\expandafter\def\csname PY@tok@mb\endcsname{\def\PY@tc##1{\textcolor[rgb]{0.40,0.40,0.40}{##1}}}
\expandafter\def\csname PY@tok@ss\endcsname{\def\PY@tc##1{\textcolor[rgb]{0.10,0.09,0.49}{##1}}}
\expandafter\def\csname PY@tok@sr\endcsname{\def\PY@tc##1{\textcolor[rgb]{0.73,0.40,0.53}{##1}}}
\expandafter\def\csname PY@tok@mo\endcsname{\def\PY@tc##1{\textcolor[rgb]{0.40,0.40,0.40}{##1}}}
\expandafter\def\csname PY@tok@kd\endcsname{\let\PY@bf=\textbf\def\PY@tc##1{\textcolor[rgb]{0.00,0.50,0.00}{##1}}}
\expandafter\def\csname PY@tok@mi\endcsname{\def\PY@tc##1{\textcolor[rgb]{0.40,0.40,0.40}{##1}}}
\expandafter\def\csname PY@tok@kn\endcsname{\let\PY@bf=\textbf\def\PY@tc##1{\textcolor[rgb]{0.00,0.50,0.00}{##1}}}
\expandafter\def\csname PY@tok@cpf\endcsname{\let\PY@it=\textit\def\PY@tc##1{\textcolor[rgb]{0.25,0.50,0.50}{##1}}}
\expandafter\def\csname PY@tok@kr\endcsname{\let\PY@bf=\textbf\def\PY@tc##1{\textcolor[rgb]{0.00,0.50,0.00}{##1}}}
\expandafter\def\csname PY@tok@s\endcsname{\def\PY@tc##1{\textcolor[rgb]{0.73,0.13,0.13}{##1}}}
\expandafter\def\csname PY@tok@kp\endcsname{\def\PY@tc##1{\textcolor[rgb]{0.00,0.50,0.00}{##1}}}
\expandafter\def\csname PY@tok@w\endcsname{\def\PY@tc##1{\textcolor[rgb]{0.73,0.73,0.73}{##1}}}
\expandafter\def\csname PY@tok@kt\endcsname{\def\PY@tc##1{\textcolor[rgb]{0.69,0.00,0.25}{##1}}}
\expandafter\def\csname PY@tok@sc\endcsname{\def\PY@tc##1{\textcolor[rgb]{0.73,0.13,0.13}{##1}}}
\expandafter\def\csname PY@tok@sb\endcsname{\def\PY@tc##1{\textcolor[rgb]{0.73,0.13,0.13}{##1}}}
\expandafter\def\csname PY@tok@k\endcsname{\let\PY@bf=\textbf\def\PY@tc##1{\textcolor[rgb]{0.00,0.50,0.00}{##1}}}
\expandafter\def\csname PY@tok@se\endcsname{\let\PY@bf=\textbf\def\PY@tc##1{\textcolor[rgb]{0.73,0.40,0.13}{##1}}}
\expandafter\def\csname PY@tok@sd\endcsname{\let\PY@it=\textit\def\PY@tc##1{\textcolor[rgb]{0.73,0.13,0.13}{##1}}}

\def\PYZbs{\char`\\}
\def\PYZus{\char`\_}
\def\PYZob{\char`\{}
\def\PYZcb{\char`\}}
\def\PYZca{\char`\^}
\def\PYZam{\char`\&}
\def\PYZlt{\char`\<}
\def\PYZgt{\char`\>}
\def\PYZsh{\char`\#}
\def\PYZpc{\char`\%}
\def\PYZdl{\char`\$}
\def\PYZhy{\char`\-}
\def\PYZsq{\char`\'}
\def\PYZdq{\char`\"}
\def\PYZti{\char`\~}
% for compatibility with earlier versions
\def\PYZat{@}
\def\PYZlb{[}
\def\PYZrb{]}
\makeatother


    % Exact colors from NB
    \definecolor{incolor}{rgb}{0.0, 0.0, 0.5}
    \definecolor{outcolor}{rgb}{0.545, 0.0, 0.0}



    
    % Prevent overflowing lines due to hard-to-break entities
    \sloppy 
    % Setup hyperref package
    \hypersetup{
      breaklinks=true,  % so long urls are correctly broken across lines
      colorlinks=true,
      urlcolor=blue,
      linkcolor=darkorange,
      citecolor=darkgreen,
      }
    % Slightly bigger margins than the latex defaults
    
    \geometry{verbose,tmargin=1in,bmargin=1in,lmargin=1in,rmargin=1in}
    
    

    \begin{document}
    
    
    \maketitle
    
    

    
    \section{Machine Learning Engineer
Nanodegree}\label{machine-learning-engineer-nanodegree}

\subsection{Unsupervised Learning}\label{unsupervised-learning}

\subsection{Project 3: Creating Customer
Segments}\label{project-3-creating-customer-segments}

    \subsection{Getting Started}\label{getting-started}

In this project, you will analyze a dataset containing data on various
customers' annual spending amounts (reported in \emph{monetary units})
of diverse product categories for internal structure. One goal of this
project is to best describe the variation in the different types of
customers that a wholesale distributor interacts with. Doing so would
equip the distributor with insight into how to best structure their
delivery service to meet the needs of each customer.

The dataset for this project can be found on the
\href{https://archive.ics.uci.edu/ml/datasets/Wholesale+customers}{UCI
Machine Learning Repository}. For the purposes of this project, the
features \texttt{\textquotesingle{}Channel\textquotesingle{}} and
\texttt{\textquotesingle{}Region\textquotesingle{}} will be excluded in
the analysis --- with focus instead on the six product categories
recorded for customers.

Run the code block below to load the wholesale customers dataset, along
with a few of the necessary Python libraries required for this project.
You will know the dataset loaded successfully if the size of the dataset
is reported.

    \begin{Verbatim}[commandchars=\\\{\}]
{\color{incolor}In [{\color{incolor}1}]:} \PY{c+c1}{\PYZsh{} Import libraries necessary for this project}
        \PY{k+kn}{import} \PY{n+nn}{numpy} \PY{k+kn}{as} \PY{n+nn}{np}
        \PY{k+kn}{import} \PY{n+nn}{pandas} \PY{k+kn}{as} \PY{n+nn}{pd}
        \PY{k+kn}{import} \PY{n+nn}{renders} \PY{k+kn}{as} \PY{n+nn}{rs}
        \PY{k+kn}{import} \PY{n+nn}{matplotlib.pyplot} \PY{k+kn}{as} \PY{n+nn}{pl}
        \PY{k+kn}{from} \PY{n+nn}{IPython.display} \PY{k+kn}{import} \PY{n}{display} \PY{c+c1}{\PYZsh{} Allows the use of display() for DataFrames}
        
        \PY{c+c1}{\PYZsh{} Import seaborn without warnings}
        \PY{k+kn}{import} \PY{n+nn}{warnings}
        \PY{n}{warnings}\PY{o}{.}\PY{n}{filterwarnings}\PY{p}{(}\PY{l+s+s2}{\PYZdq{}}\PY{l+s+s2}{ignore}\PY{l+s+s2}{\PYZdq{}}\PY{p}{)}
        \PY{k+kn}{import} \PY{n+nn}{seaborn} \PY{k+kn}{as} \PY{n+nn}{sns}
        
        \PY{c+c1}{\PYZsh{} Show matplotlib plots inline (nicely formatted in the notebook)}
        \PY{o}{\PYZpc{}}\PY{k}{matplotlib} inline
        
        \PY{c+c1}{\PYZsh{} Load the wholesale customers dataset}
        \PY{k}{try}\PY{p}{:}
            \PY{n}{data} \PY{o}{=} \PY{n}{pd}\PY{o}{.}\PY{n}{read\PYZus{}csv}\PY{p}{(}\PY{l+s+s2}{\PYZdq{}}\PY{l+s+s2}{customers.csv}\PY{l+s+s2}{\PYZdq{}}\PY{p}{)}
            \PY{n}{data}\PY{o}{.}\PY{n}{drop}\PY{p}{(}\PY{p}{[}\PY{l+s+s1}{\PYZsq{}}\PY{l+s+s1}{Region}\PY{l+s+s1}{\PYZsq{}}\PY{p}{,} \PY{l+s+s1}{\PYZsq{}}\PY{l+s+s1}{Channel}\PY{l+s+s1}{\PYZsq{}}\PY{p}{]}\PY{p}{,} \PY{n}{axis} \PY{o}{=} \PY{l+m+mi}{1}\PY{p}{,} \PY{n}{inplace} \PY{o}{=} \PY{n+nb+bp}{True}\PY{p}{)}
            \PY{k}{print} \PY{l+s+s2}{\PYZdq{}}\PY{l+s+s2}{Wholesale customers dataset has \PYZob{}\PYZcb{} samples with \PYZob{}\PYZcb{} features each.}\PY{l+s+s2}{\PYZdq{}}\PY{o}{.}\PY{n}{format}\PY{p}{(}\PY{o}{*}\PY{n}{data}\PY{o}{.}\PY{n}{shape}\PY{p}{)}
        \PY{k}{except}\PY{p}{:}
            \PY{k}{print} \PY{l+s+s2}{\PYZdq{}}\PY{l+s+s2}{Dataset could not be loaded. Is the dataset missing?}\PY{l+s+s2}{\PYZdq{}}
\end{Verbatim}

    \begin{Verbatim}[commandchars=\\\{\}]
Wholesale customers dataset has 440 samples with 6 features each.
    \end{Verbatim}

    \subsection{Data Exploration}\label{data-exploration}

In this section, you will begin exploring the data through
visualizations and code to understand how each feature is related to the
others. You will observe a statistical description of the dataset,
consider the relevance of each feature, and select a few sample data
points from the dataset which you will track through the course of this
project.

Run the code block below to observe a statistical description of the
dataset. Note that the dataset is composed of six important product
categories: \textbf{`Fresh'}, \textbf{`Milk'}, \textbf{`Grocery'},
\textbf{`Frozen'}, \textbf{`Detergents\_Paper'}, and
\textbf{`Delicatessen'}. Consider what each category represents in terms
of products you could purchase.

    \begin{Verbatim}[commandchars=\\\{\}]
{\color{incolor}In [{\color{incolor}2}]:} \PY{c+c1}{\PYZsh{} Display a description of the dataset}
        \PY{n}{display}\PY{p}{(}\PY{n}{data}\PY{o}{.}\PY{n}{describe}\PY{p}{(}\PY{p}{)}\PY{p}{)}
\end{Verbatim}

    
    \begin{verbatim}
               Fresh          Milk       Grocery        Frozen  \
count     440.000000    440.000000    440.000000    440.000000   
mean    12000.297727   5796.265909   7951.277273   3071.931818   
std     12647.328865   7380.377175   9503.162829   4854.673333   
min         3.000000     55.000000      3.000000     25.000000   
25%      3127.750000   1533.000000   2153.000000    742.250000   
50%      8504.000000   3627.000000   4755.500000   1526.000000   
75%     16933.750000   7190.250000  10655.750000   3554.250000   
max    112151.000000  73498.000000  92780.000000  60869.000000   

       Detergents_Paper  Delicatessen  
count        440.000000    440.000000  
mean        2881.493182   1524.870455  
std         4767.854448   2820.105937  
min            3.000000      3.000000  
25%          256.750000    408.250000  
50%          816.500000    965.500000  
75%         3922.000000   1820.250000  
max        40827.000000  47943.000000  
    \end{verbatim}

    
    \subsubsection{Implementation: Selecting
Samples}\label{implementation-selecting-samples}

To get a better understanding of the customers and how their data will
transform through the analysis, it would be best to select a few sample
data points and explore them in more detail. In the code block below,
add \textbf{three} indices of your choice to the \texttt{indices} list
which will represent the customers to track. It is suggested to try
different sets of samples until you obtain customers that vary
significantly from one another.

    \begin{Verbatim}[commandchars=\\\{\}]
{\color{incolor}In [{\color{incolor}3}]:} \PY{c+c1}{\PYZsh{} Select three indices to sample from the dataset}
        \PY{n}{indices} \PY{o}{=} \PY{p}{[}\PY{l+m+mi}{100}\PY{p}{,} \PY{l+m+mi}{198}\PY{p}{,} \PY{l+m+mi}{305}\PY{p}{]}
        
        \PY{c+c1}{\PYZsh{} Create a DataFrame of the chosen samples}
        \PY{n}{samples} \PY{o}{=} \PY{n}{pd}\PY{o}{.}\PY{n}{DataFrame}\PY{p}{(}\PY{n}{data}\PY{o}{.}\PY{n}{loc}\PY{p}{[}\PY{n}{indices}\PY{p}{]}\PY{p}{,} \PY{n}{columns} \PY{o}{=} \PY{n}{data}\PY{o}{.}\PY{n}{keys}\PY{p}{(}\PY{p}{)}\PY{p}{)}\PY{o}{.}\PY{n}{reset\PYZus{}index}\PY{p}{(}\PY{n}{drop} \PY{o}{=} \PY{n+nb+bp}{True}\PY{p}{)}
        \PY{k}{print} \PY{l+s+s2}{\PYZdq{}}\PY{l+s+s2}{Chosen samples of wholesale customers dataset:}\PY{l+s+s2}{\PYZdq{}}
        \PY{n}{display}\PY{p}{(}\PY{n}{samples}\PY{p}{)}
\end{Verbatim}

    \begin{Verbatim}[commandchars=\\\{\}]
Chosen samples of wholesale customers dataset:
    \end{Verbatim}

    
    \begin{verbatim}
   Fresh   Milk  Grocery  Frozen  Detergents_Paper  Delicatessen
0  11594   7779    12144    3252              8035          3029
1  11686   2154     6824    3527               592           697
2    243  12939     8852     799              3909           211
    \end{verbatim}

    
    \paragraph{Additional:}\label{additional}

    \begin{Verbatim}[commandchars=\\\{\}]
{\color{incolor}In [{\color{incolor}4}]:} \PY{c+c1}{\PYZsh{} Make heatmap based on percentile distributions of each sample}
        \PY{n}{pcts} \PY{o}{=} \PY{l+m+mf}{100.} \PY{o}{*} \PY{n}{data}\PY{o}{.}\PY{n}{rank}\PY{p}{(}\PY{n}{axis}\PY{o}{=}\PY{l+m+mi}{0}\PY{p}{,} \PY{n}{pct}\PY{o}{=}\PY{n+nb+bp}{True}\PY{p}{)}\PY{o}{.}\PY{n}{iloc}\PY{p}{[}\PY{n}{indices}\PY{p}{]}\PY{o}{.}\PY{n}{round}\PY{p}{(}\PY{n}{decimals}\PY{o}{=}\PY{l+m+mi}{3}\PY{p}{)}
        \PY{k}{print} \PY{n}{pcts}
        
        \PY{n}{pl}\PY{o}{.}\PY{n}{figure}\PY{p}{(}\PY{n}{figsize}\PY{o}{=}\PY{p}{(}\PY{l+m+mi}{12}\PY{p}{,}\PY{l+m+mi}{8}\PY{p}{)}\PY{p}{)}
        \PY{n}{sns}\PY{o}{.}\PY{n}{heatmap}\PY{p}{(}\PY{n}{pcts}\PY{o}{.}\PY{n}{reset\PYZus{}index}\PY{p}{(}\PY{n}{drop}\PY{o}{=}\PY{n+nb+bp}{True}\PY{p}{)}\PY{p}{,} \PY{n}{annot}\PY{o}{=}\PY{n+nb+bp}{True}\PY{p}{,} \PY{n}{vmin}\PY{o}{=}\PY{l+m+mi}{1}\PY{p}{,} \PY{n}{vmax}\PY{o}{=}\PY{l+m+mi}{99}\PY{p}{,} \PY{n}{fmt}\PY{o}{=}\PY{l+s+s1}{\PYZsq{}}\PY{l+s+s1}{.1f}\PY{l+s+s1}{\PYZsq{}}\PY{p}{,} \PY{n}{cmap}\PY{o}{=}\PY{l+s+s1}{\PYZsq{}}\PY{l+s+s1}{YlGnBu}\PY{l+s+s1}{\PYZsq{}}\PY{p}{)}
        \PY{n}{pl}\PY{o}{.}\PY{n}{title}\PY{p}{(}\PY{l+s+s1}{\PYZsq{}}\PY{l+s+s1}{Percentile ranks of}\PY{l+s+se}{\PYZbs{}n}\PY{l+s+s1}{samples}\PY{l+s+se}{\PYZbs{}\PYZsq{}}\PY{l+s+s1}{ category spending}\PY{l+s+s1}{\PYZsq{}}\PY{p}{)}
        \PY{n}{pl}\PY{o}{.}\PY{n}{xticks}\PY{p}{(}\PY{n}{rotation}\PY{o}{=}\PY{l+m+mi}{45}\PY{p}{,} \PY{n}{ha}\PY{o}{=}\PY{l+s+s1}{\PYZsq{}}\PY{l+s+s1}{center}\PY{l+s+s1}{\PYZsq{}}\PY{p}{)}\PY{p}{;}
\end{Verbatim}

    \begin{Verbatim}[commandchars=\\\{\}]
Fresh  Milk  Grocery  Frozen  Detergents\_Paper  Delicatessen
100   62.5  78.6     80.2    73.0              91.4          90.5
198   63.0  34.8     60.9    74.8              43.4          38.9
305    3.6  91.6     69.5    27.0              75.0          11.6
    \end{Verbatim}

    \begin{center}
    \adjustimage{max size={0.9\linewidth}{0.9\paperheight}}{customer_segments_files/customer_segments_8_1.png}
    \end{center}
    { \hspace*{\fill} \\}
    
    \subsubsection{Question 1}\label{question-1}

Consider the total purchase cost of each product category and the
statistical description of the dataset above for your sample
customers.\\
\emph{What kind of establishment (customer) could each of the three
samples you've chosen represent?}\\
\textbf{Hint:} Examples of establishments include places like markets,
cafes, and retailers, among many others. Avoid using names for
establishments, such as saying \emph{``McDonalds''} when describing a
sample customer as a restaurant.

    \textbf{Answer:}

Considering the statistical description of the dataset (type of
products), and subjectively judging on relative proportion of each type
for sample customers: 1. Sample no. 100 (Sample 0 in table) has total
purchase cost for `Milk' and `Grocery' slightly above 75th percentile,
and `Detergents\_Paper' and `Delicatessen' squarely falling in the last
quartile of their respective categories. The other two product
categories lie approximately in the 3rd quartile. So relatively, it
appears this customer is a high consumer of detergent products and fine
food products, followed by the other categories, which might imply that
the type of business associated could be a hotel. Considering the volume
of all product categories, it maybe further inferred that this customer
could be a medium(-to-large) scale hotel. 2. Sample no. 198 (Sample 1 in
table) has the highest total purchase costs for `Fresh', `Frozen' and
`Grocery', all of them falling in the third quartiles of their
respective categories, followed by `Milk', `Detergents\_Paper' and
`Delicatessen' falling in the second quartile. The relative proportions
including the high consumption of fresh, frozen and grocery products and
the total volume of each product category might indicate that this data
point could represent a small-scale retailer or departmental store. 3.
Sample no. 305 (Sample 2 in table) has the highest total purchase cost
for `Milk', which lies in the fourth quartile, followed by
`Detergents\_Paper' and `Grocery' falling in their third quartiles, and
finally trailed by the other categories in their first quartiles. The
high consumption of milk products, and relatively smaller consumption of
other products might indicate that this data point could be a
small(-to-medium) scale cafe.

    \subsubsection{Implementation: Feature
Relevance}\label{implementation-feature-relevance}

One interesting thought to consider is if one (or more) of the six
product categories is actually relevant for understanding customer
purchasing. That is to say, is it possible to determine whether
customers purchasing some amount of one category of products will
necessarily purchase some proportional amount of another category of
products? We can make this determination quite easily by training a
supervised regression learner on a subset of the data with one feature
removed, and then score how well that model can predict the removed
feature.

In the code block below, you will need to implement the following: -
Assign \texttt{new\_data} a copy of the data by removing a feature of
your choice using the \texttt{DataFrame.drop} function. - Use
\texttt{sklearn.cross\_validation.train\_test\_split} to split the
dataset into training and testing sets. - Use the removed feature as
your target label. Set a \texttt{test\_size} of \texttt{0.25} and set a
\texttt{random\_state}. - Import a decision tree regressor, set a
\texttt{random\_state}, and fit the learner to the training data. -
Report the prediction score of the testing set using the regressor's
\texttt{score} function.

    \begin{Verbatim}[commandchars=\\\{\}]
{\color{incolor}In [{\color{incolor}5}]:} \PY{k+kn}{from} \PY{n+nn}{sklearn.cross\PYZus{}validation} \PY{k+kn}{import} \PY{n}{train\PYZus{}test\PYZus{}split}
        \PY{k+kn}{from} \PY{n+nn}{sklearn.tree} \PY{k+kn}{import} \PY{n}{DecisionTreeRegressor}
        
        \PY{c+c1}{\PYZsh{} Required test data size}
        \PY{n}{test\PYZus{}size} \PY{o}{=} \PY{l+m+mf}{0.25}
        
        \PY{k}{for} \PY{n}{feature\PYZus{}to\PYZus{}predict} \PY{o+ow}{in} \PY{n}{data}\PY{o}{.}\PY{n}{columns}\PY{p}{:}    
            \PY{n}{new\PYZus{}data} \PY{o}{=} \PY{n}{data}\PY{o}{.}\PY{n}{copy}\PY{p}{(}\PY{p}{)}
            \PY{n}{new\PYZus{}labels} \PY{o}{=} \PY{n}{new\PYZus{}data}\PY{p}{[}\PY{n}{feature\PYZus{}to\PYZus{}predict}\PY{p}{]}
            \PY{n}{new\PYZus{}data}\PY{o}{.}\PY{n}{drop}\PY{p}{(}\PY{p}{[}\PY{n}{feature\PYZus{}to\PYZus{}predict}\PY{p}{]}\PY{p}{,} \PY{n}{axis}\PY{o}{=}\PY{l+m+mi}{1}\PY{p}{,} \PY{n}{inplace}\PY{o}{=}\PY{n+nb+bp}{True}\PY{p}{)}
        
            \PY{c+c1}{\PYZsh{} Split the data into training and testing sets using the given feature as the target  }
            \PY{n}{random\PYZus{}state} \PY{o}{=} \PY{n}{np}\PY{o}{.}\PY{n}{random}\PY{o}{.}\PY{n}{RandomState}\PY{p}{(}\PY{l+m+mi}{101010}\PY{p}{)}
            \PY{n}{X\PYZus{}train}\PY{p}{,} \PY{n}{X\PYZus{}test}\PY{p}{,} \PY{n}{y\PYZus{}train}\PY{p}{,} \PY{n}{y\PYZus{}test} \PY{o}{=} \PY{n}{train\PYZus{}test\PYZus{}split}\PY{p}{(}
                \PY{n}{new\PYZus{}data}\PY{p}{,} \PY{n}{new\PYZus{}labels}\PY{p}{,} \PY{n}{test\PYZus{}size}\PY{o}{=}\PY{n}{test\PYZus{}size}\PY{p}{,} \PY{n}{random\PYZus{}state}\PY{o}{=}\PY{n}{random\PYZus{}state}\PY{p}{)}
        
            \PY{c+c1}{\PYZsh{} Create a decision tree regressor and fit it to the training set    }
            \PY{n}{regressor} \PY{o}{=} \PY{n}{DecisionTreeRegressor}\PY{p}{(}\PY{n}{random\PYZus{}state}\PY{o}{=}\PY{n}{random\PYZus{}state}\PY{p}{)} \PY{c+c1}{\PYZsh{} Default params should be OK for this test}
            \PY{n}{regressor}\PY{o}{.}\PY{n}{fit}\PY{p}{(}\PY{n}{X\PYZus{}train}\PY{p}{,} \PY{n}{y\PYZus{}train}\PY{p}{)}
        
            \PY{c+c1}{\PYZsh{} Report the score of the prediction using the testing set}
            \PY{k+kn}{from} \PY{n+nn}{sklearn.metrics} \PY{k+kn}{import} \PY{n}{r2\PYZus{}score}
            \PY{n}{y\PYZus{}pred} \PY{o}{=} \PY{n}{regressor}\PY{o}{.}\PY{n}{predict}\PY{p}{(}\PY{n}{X\PYZus{}test}\PY{p}{)}
            \PY{n}{score} \PY{o}{=} \PY{n}{r2\PYZus{}score}\PY{p}{(}\PY{n}{y\PYZus{}test}\PY{p}{,} \PY{n}{y\PYZus{}pred}\PY{p}{)}
            \PY{k}{print} \PY{l+s+s2}{\PYZdq{}}\PY{l+s+s2}{R2 score for predicting [\PYZob{}\PYZcb{}] feature is \PYZob{}\PYZcb{}}\PY{l+s+s2}{\PYZdq{}}\PY{o}{.}\PY{n}{format}\PY{p}{(}\PY{n}{feature\PYZus{}to\PYZus{}predict}\PY{p}{,} \PY{n}{score}\PY{p}{)}
            
        \PY{c+c1}{\PYZsh{} Random state to be used by all further functions for the project}
        \PY{n}{random\PYZus{}state} \PY{o}{=} \PY{n}{np}\PY{o}{.}\PY{n}{random}\PY{o}{.}\PY{n}{RandomState}\PY{p}{(}\PY{l+m+mi}{101010}\PY{p}{)} \PY{c+c1}{\PYZsh{} ten\PYZhy{}ten\PYZhy{}ten}
\end{Verbatim}

    \begin{Verbatim}[commandchars=\\\{\}]
R2 score for predicting [Fresh] feature is -1.18721250685
R2 score for predicting [Milk] feature is -1.06747225492
R2 score for predicting [Grocery] feature is 0.636420923989
R2 score for predicting [Frozen] feature is -2.82322335641
R2 score for predicting [Detergents\_Paper] feature is 0.623542625285
R2 score for predicting [Delicatessen] feature is -2.40262777355
    \end{Verbatim}

    \subsubsection{Question 2}\label{question-2}

\emph{Which feature did you attempt to predict? What was the reported
prediction score? Is this feature relevant for identifying a specific
customer?}\\
\textbf{Hint:} The coefficient of determination, \texttt{R\^{}2}, is
scored between 0 and 1, with 1 being a perfect fit. A negative
\texttt{R\^{}2} implies the model fails to fit the data.

    \textbf{Answer:}

A test of feature relevance is tried for all features in the dataset. It
looks like Fresh, Milk, Frozen and Delicatessen features have, in
general, very less correlation with other features and so they cannot be
predicted directly as a function of other features; they all have
negative R\^{}2 scores. Whereas, there seems to be some correlation with
other features of dataset and/or correlation between Grocery and
Detergents\_Paper features. This is indicated by their positive R\^{}2
scores of around 0.6.

So, since Fresh, Milk, Frozen and Delicatessen are heavily uncorrelated
with other features, they will be valuable to identify specific customer
groups. On the other hand, the features Grocery and Detergents\_Paper
individually will contribute lesser to identifying specific groups as
they seem to possess some redundant information when dataset is
considered as a whole.

    \subsubsection{Visualize Feature
Distributions}\label{visualize-feature-distributions}

To get a better understanding of the dataset, we can construct a scatter
matrix of each of the six product features present in the data. If you
found that the feature you attempted to predict above is relevant for
identifying a specific customer, then the scatter matrix below may not
show any correlation between that feature and the others. Conversely, if
you believe that feature is not relevant for identifying a specific
customer, the scatter matrix might show a correlation between that
feature and another feature in the data. Run the code block below to
produce a scatter matrix.

    \begin{Verbatim}[commandchars=\\\{\}]
{\color{incolor}In [{\color{incolor}6}]:} \PY{c+c1}{\PYZsh{} Produce a scatter matrix for each pair of features in the data}
        \PY{n}{pd}\PY{o}{.}\PY{n}{scatter\PYZus{}matrix}\PY{p}{(}\PY{n}{data}\PY{p}{,} \PY{n}{alpha} \PY{o}{=} \PY{l+m+mf}{0.3}\PY{p}{,} \PY{n}{figsize} \PY{o}{=} \PY{p}{(}\PY{l+m+mi}{14}\PY{p}{,}\PY{l+m+mi}{8}\PY{p}{)}\PY{p}{,} \PY{n}{diagonal} \PY{o}{=} \PY{l+s+s1}{\PYZsq{}}\PY{l+s+s1}{kde}\PY{l+s+s1}{\PYZsq{}}\PY{p}{)}\PY{p}{;}
\end{Verbatim}

    \begin{center}
    \adjustimage{max size={0.9\linewidth}{0.9\paperheight}}{customer_segments_files/customer_segments_16_0.png}
    \end{center}
    { \hspace*{\fill} \\}
    
    \paragraph{Additional:}\label{additional}

    \begin{Verbatim}[commandchars=\\\{\}]
{\color{incolor}In [{\color{incolor}7}]:} \PY{c+c1}{\PYZsh{} Construct heatmap of cross\PYZhy{}correlation matrix}
        \PY{n}{corr} \PY{o}{=} \PY{n}{data}\PY{o}{.}\PY{n}{corr}\PY{p}{(}\PY{p}{)}
        \PY{n}{mask} \PY{o}{=} \PY{n}{np}\PY{o}{.}\PY{n}{zeros\PYZus{}like}\PY{p}{(}\PY{n}{corr}\PY{p}{)}
        \PY{n}{mask}\PY{p}{[}\PY{n}{np}\PY{o}{.}\PY{n}{triu\PYZus{}indices\PYZus{}from}\PY{p}{(}\PY{n}{mask}\PY{p}{,} \PY{l+m+mi}{1}\PY{p}{)}\PY{p}{]} \PY{o}{=} \PY{n+nb+bp}{True}
        \PY{k}{with} \PY{n}{sns}\PY{o}{.}\PY{n}{axes\PYZus{}style}\PY{p}{(}\PY{l+s+s2}{\PYZdq{}}\PY{l+s+s2}{white}\PY{l+s+s2}{\PYZdq{}}\PY{p}{)}\PY{p}{:}
            \PY{n}{pl}\PY{o}{.}\PY{n}{figure}\PY{p}{(}\PY{n}{figsize}\PY{o}{=}\PY{p}{(}\PY{l+m+mi}{8}\PY{p}{,}\PY{l+m+mi}{8}\PY{p}{)}\PY{p}{)}
            \PY{n}{ax} \PY{o}{=} \PY{n}{sns}\PY{o}{.}\PY{n}{heatmap}\PY{p}{(}\PY{n}{corr}\PY{p}{,} \PY{n}{mask}\PY{o}{=}\PY{n}{mask}\PY{p}{,} \PY{n}{square}\PY{o}{=}\PY{n+nb+bp}{True}\PY{p}{,} \PY{n}{cmap}\PY{o}{=}\PY{l+s+s1}{\PYZsq{}}\PY{l+s+s1}{RdBu\PYZus{}r}\PY{l+s+s1}{\PYZsq{}}\PY{p}{,} \PY{n}{annot}\PY{o}{=}\PY{n+nb+bp}{True}\PY{p}{)}
            \PY{n}{pl}\PY{o}{.}\PY{n}{xticks}\PY{p}{(}\PY{n}{rotation}\PY{o}{=}\PY{l+m+mi}{45}\PY{p}{,} \PY{n}{ha}\PY{o}{=}\PY{l+s+s1}{\PYZsq{}}\PY{l+s+s1}{center}\PY{l+s+s1}{\PYZsq{}}\PY{p}{)}
            \PY{n}{pl}\PY{o}{.}\PY{n}{yticks}\PY{p}{(}\PY{n}{rotation}\PY{o}{=}\PY{l+m+mi}{45}\PY{p}{,} \PY{n}{va}\PY{o}{=}\PY{l+s+s1}{\PYZsq{}}\PY{l+s+s1}{center}\PY{l+s+s1}{\PYZsq{}}\PY{p}{)}
\end{Verbatim}

    \begin{center}
    \adjustimage{max size={0.9\linewidth}{0.9\paperheight}}{customer_segments_files/customer_segments_18_0.png}
    \end{center}
    { \hspace*{\fill} \\}
    
    \subsubsection{Question 3}\label{question-3}

\emph{Are there any pairs of features which exhibit some degree of
correlation? Does this confirm or deny your suspicions about the
relevance of the feature you attempted to predict? How is the data for
those features distributed?}\\
\textbf{Hint:} Is the data normally distributed? Where do most of the
data points lie?

    \textbf{Answer:}

From the scatter plot, it is evident that there is some correlation
between `Grocery' and `Detergents\_Paper' features. The general shape of
\{`Grocery', `Detergents\_Paper'\} and \{`Detergents\_Paper',
`Grocery'\} scatter plots are thin and elongated. That is, there is very
less (co-)variance between the pair of features. \{`Milk', `Grocery'\}
and \{`Milk', `Detergents\_Paper'\} (and their inverses) too exhibit
slight elongation, but probably not as immediately apparent as `Grocery'
and `Detergents\_Paper' pair. This confirms the initial suspicions in
the ``Feature relevance'' section that one of the features when
considered individually doesn't add much information to the dataset and
that one feature was likely heavily used to predict the other. In
simpler words, `Grocery' (label) was being predicted mainly with the
information from `Detergents\_Paper' (feature) and vice-versa in the
regressor example above.

The data points don't follow a Gaussian distribution, mainly because the
median of each feature doesn't seem to co-incide with the respective
means. Furthermore, a nice symmetric bell curve is not apparent in the
diagonal of the scatter plot matrix (although it could be symmetric bell
shaped, if not for the cut-off point below which no data points exist.)
So in effect, the correlated features \{`Grocery', `Detergents\_Paper'\}
too don't have Gaussian distributions, with their means and medians
quite separated.

    \subsection{Data Preprocessing}\label{data-preprocessing}

In this section, you will preprocess the data to create a better
representation of customers by performing a scaling on the data and
detecting (and optionally removing) outliers. Preprocessing data is
often times a critical step in assuring that results you obtain from
your analysis are significant and meaningful.

    \subsubsection{Implementation: Feature
Scaling}\label{implementation-feature-scaling}

If data is not normally distributed, especially if the mean and median
vary significantly (indicating a large skew), it is most
\href{http://econbrowser.com/archives/2014/02/use-of-logarithms-in-economics}{often
appropriate} to apply a non-linear scaling --- particularly for
financial data. One way to achieve this scaling is by using a
\href{http://scipy.github.io/devdocs/generated/scipy.stats.boxcox.html}{Box-Cox
test}, which calculates the best power transformation of the data that
reduces skewness. A simpler approach which can work in most cases would
be applying the natural logarithm.

In the code block below, you will need to implement the following: -
Assign a copy of the data to \texttt{log\_data} after applying a
logarithm scaling. Use the \texttt{np.log} function for this. - Assign a
copy of the sample data to \texttt{log\_samples} after applying a
logrithm scaling. Again, use \texttt{np.log}.

    \begin{Verbatim}[commandchars=\\\{\}]
{\color{incolor}In [{\color{incolor}8}]:} \PY{c+c1}{\PYZsh{} Scale the data using the natural logarithm}
        \PY{n}{log\PYZus{}data} \PY{o}{=} \PY{n}{np}\PY{o}{.}\PY{n}{log}\PY{p}{(}\PY{n}{data}\PY{p}{)}
        
        \PY{c+c1}{\PYZsh{} Scale the sample data using the natural logarithm}
        \PY{n}{log\PYZus{}samples} \PY{o}{=} \PY{n}{np}\PY{o}{.}\PY{n}{log}\PY{p}{(}\PY{n}{samples}\PY{p}{)}
        
        \PY{c+c1}{\PYZsh{} Produce a scatter matrix for each pair of newly\PYZhy{}transformed features}
        \PY{n}{axes} \PY{o}{=} \PY{n}{pd}\PY{o}{.}\PY{n}{scatter\PYZus{}matrix}\PY{p}{(}\PY{n}{log\PYZus{}data}\PY{p}{,} \PY{n}{alpha} \PY{o}{=} \PY{l+m+mf}{0.3}\PY{p}{,} \PY{n}{figsize} \PY{o}{=} \PY{p}{(}\PY{l+m+mi}{14}\PY{p}{,}\PY{l+m+mi}{8}\PY{p}{)}\PY{p}{,} \PY{n}{diagonal} \PY{o}{=} \PY{l+s+s1}{\PYZsq{}}\PY{l+s+s1}{kde}\PY{l+s+s1}{\PYZsq{}}\PY{p}{)}\PY{p}{;}
        
        \PY{c+c1}{\PYZsh{} *Additional*}
        \PY{c+c1}{\PYZsh{} Annotate scatter matrix with correlations}
        \PY{n}{corr} \PY{o}{=} \PY{n}{log\PYZus{}data}\PY{o}{.}\PY{n}{corr}\PY{p}{(}\PY{p}{)}\PY{o}{.}\PY{n}{as\PYZus{}matrix}\PY{p}{(}\PY{p}{)}
        \PY{k}{for} \PY{n}{i}\PY{p}{,} \PY{n}{j} \PY{o+ow}{in} \PY{n+nb}{zip}\PY{p}{(}\PY{o}{*}\PY{n}{pl}\PY{o}{.}\PY{n}{np}\PY{o}{.}\PY{n}{triu\PYZus{}indices\PYZus{}from}\PY{p}{(}\PY{n}{axes}\PY{p}{,} \PY{n}{k}\PY{o}{=}\PY{l+m+mi}{1}\PY{p}{)}\PY{p}{)}\PY{p}{:}
            \PY{n}{axes}\PY{p}{[}\PY{n}{i}\PY{p}{,}\PY{n}{j}\PY{p}{]}\PY{o}{.}\PY{n}{annotate}\PY{p}{(}\PY{l+s+s2}{\PYZdq{}}\PY{l+s+si}{\PYZpc{}+.3f}\PY{l+s+s2}{\PYZdq{}} \PY{o}{\PYZpc{}}\PY{k}{corr}[i,j], (0.8, 0.2), xycoords=\PYZsq{}axes fraction\PYZsq{}, 
                               \PY{n}{ha}\PY{o}{=}\PY{l+s+s1}{\PYZsq{}}\PY{l+s+s1}{center}\PY{l+s+s1}{\PYZsq{}}\PY{p}{,} \PY{n}{va}\PY{o}{=}\PY{l+s+s1}{\PYZsq{}}\PY{l+s+s1}{center}\PY{l+s+s1}{\PYZsq{}}\PY{p}{,}\PY{n}{color}\PY{o}{=}\PY{l+s+s2}{\PYZdq{}}\PY{l+s+s2}{red}\PY{l+s+s2}{\PYZdq{}}\PY{p}{,} \PY{n}{fontsize}\PY{o}{=}\PY{l+m+mi}{14}\PY{p}{)}
\end{Verbatim}

    \begin{center}
    \adjustimage{max size={0.9\linewidth}{0.9\paperheight}}{customer_segments_files/customer_segments_23_0.png}
    \end{center}
    { \hspace*{\fill} \\}
    
    \subsubsection{Observation}\label{observation}

After applying a natural logarithm scaling to the data, the distribution
of each feature should appear much more normal. For any pairs of
features you may have identified earlier as being correlated, observe
here whether that correlation is still present (and whether it is now
stronger or weaker than before).

Run the code below to see how the sample data has changed after having
the natural logarithm applied to it.

    \begin{Verbatim}[commandchars=\\\{\}]
{\color{incolor}In [{\color{incolor}9}]:} \PY{c+c1}{\PYZsh{} Display the log\PYZhy{}transformed sample data}
        \PY{n}{display}\PY{p}{(}\PY{n}{log\PYZus{}samples}\PY{p}{)}
\end{Verbatim}

    
    \begin{verbatim}
      Fresh      Milk   Grocery    Frozen  Detergents_Paper  Delicatessen
0  9.358243  8.959183  9.404590  8.087025          8.991562      8.015988
1  9.366147  7.675082  8.828201  8.168203          6.383507      6.546785
2  5.493061  9.468001  9.088399  6.683361          8.271037      5.351858
    \end{verbatim}

    
    \subsubsection{Implementation: Outlier
Detection}\label{implementation-outlier-detection}

Detecting outliers in the data is extremely important in the data
preprocessing step of any analysis. The presence of outliers can often
skew results which take into consideration these data points. There are
many ``rules of thumb'' for what constitutes an outlier in a dataset.
Here, we will use
\href{http://datapigtechnologies.com/blog/index.php/highlighting-outliers-in-your-data-with-the-tukey-method/}{Tukey's
Method for identfying outliers}: An \emph{outlier step} is calculated as
1.5 times the interquartile range (IQR). A data point with a feature
that is beyond an outlier step outside of the IQR for that feature is
considered abnormal.

In the code block below, you will need to implement the following: -
Assign the value of the 25th percentile for the given feature to
\texttt{Q1}. Use \texttt{np.percentile} for this. - Assign the value of
the 75th percentile for the given feature to \texttt{Q3}. Again, use
\texttt{np.percentile}. - Assign the calculation of an outlier step for
the given feature to \texttt{step}. - Optionally remove data points from
the dataset by adding indices to the \texttt{outliers} list.

\textbf{NOTE:} If you choose to remove any outliers, ensure that the
sample data does not contain any of these points!\\
Once you have performed this implementation, the dataset will be stored
in the variable \texttt{good\_data}.

    \begin{Verbatim}[commandchars=\\\{\}]
{\color{incolor}In [{\color{incolor}10}]:} \PY{c+c1}{\PYZsh{} Describe the log\PYZus{}data so that it may be used as reference to manually inspect outliers}
         \PY{k}{print} \PY{l+s+s2}{\PYZdq{}}\PY{l+s+s2}{Description of the log transformed dataset:}\PY{l+s+s2}{\PYZdq{}}
         \PY{n}{display}\PY{p}{(}\PY{n}{log\PYZus{}data}\PY{o}{.}\PY{n}{describe}\PY{p}{(}\PY{p}{)}\PY{p}{)}
         
         \PY{n}{outlier\PYZus{}indices} \PY{o}{=} \PY{n+nb}{dict}\PY{p}{(}\PY{p}{)}
         
         \PY{c+c1}{\PYZsh{} For each feature find the data points with extreme high or low values}
         \PY{k}{for} \PY{n}{feature} \PY{o+ow}{in} \PY{n}{log\PYZus{}data}\PY{o}{.}\PY{n}{keys}\PY{p}{(}\PY{p}{)}\PY{p}{:}
             
             \PY{c+c1}{\PYZsh{} Calculate Q1 (25th percentile of the data) for the given feature}
             \PY{n}{Q1} \PY{o}{=} \PY{n}{np}\PY{o}{.}\PY{n}{percentile}\PY{p}{(}\PY{n}{log\PYZus{}data}\PY{p}{[}\PY{n}{feature}\PY{p}{]}\PY{p}{,} \PY{l+m+mi}{25}\PY{p}{)}
             
             \PY{c+c1}{\PYZsh{} Calculate Q3 (75th percentile of the data) for the given feature}
             \PY{n}{Q3} \PY{o}{=} \PY{n}{np}\PY{o}{.}\PY{n}{percentile}\PY{p}{(}\PY{n}{log\PYZus{}data}\PY{p}{[}\PY{n}{feature}\PY{p}{]}\PY{p}{,} \PY{l+m+mi}{75}\PY{p}{)}
             
             \PY{c+c1}{\PYZsh{} Use the interquartile range to calculate an outlier step (1.5 times the interquartile range)}
             \PY{n}{step} \PY{o}{=} \PY{p}{(}\PY{n}{Q3} \PY{o}{\PYZhy{}} \PY{n}{Q1}\PY{p}{)} \PY{o}{*} \PY{l+m+mf}{1.5}
             
             \PY{c+c1}{\PYZsh{} Display the outliers}
             \PY{k}{print} \PY{l+s+s2}{\PYZdq{}}\PY{l+s+s2}{Data points considered outliers for the feature }\PY{l+s+s2}{\PYZsq{}}\PY{l+s+s2}{\PYZob{}\PYZcb{}}\PY{l+s+s2}{\PYZsq{}}\PY{l+s+s2}{:}\PY{l+s+s2}{\PYZdq{}}\PY{o}{.}\PY{n}{format}\PY{p}{(}\PY{n}{feature}\PY{p}{)}
             \PY{n}{display}\PY{p}{(}\PY{n}{log\PYZus{}data}\PY{p}{[}\PY{o}{\PYZti{}}\PY{p}{(}\PY{p}{(}\PY{n}{log\PYZus{}data}\PY{p}{[}\PY{n}{feature}\PY{p}{]} \PY{o}{\PYZgt{}}\PY{o}{=} \PY{n}{Q1} \PY{o}{\PYZhy{}} \PY{n}{step}\PY{p}{)} \PY{o}{\PYZam{}} \PY{p}{(}\PY{n}{log\PYZus{}data}\PY{p}{[}\PY{n}{feature}\PY{p}{]} \PY{o}{\PYZlt{}}\PY{o}{=} \PY{n}{Q3} \PY{o}{+} \PY{n}{step}\PY{p}{)}\PY{p}{)}\PY{p}{]}\PY{p}{)}
             
             \PY{c+c1}{\PYZsh{} Count indices of outliers}
             \PY{k}{for} \PY{n}{entry} \PY{o+ow}{in} \PY{n}{log\PYZus{}data}\PY{p}{[}\PY{o}{\PYZti{}}\PY{p}{(}\PY{p}{(}\PY{n}{log\PYZus{}data}\PY{p}{[}\PY{n}{feature}\PY{p}{]} \PY{o}{\PYZgt{}}\PY{o}{=} \PY{n}{Q1} \PY{o}{\PYZhy{}} \PY{n}{step}\PY{p}{)} \PY{o}{\PYZam{}} \PY{p}{(}\PY{n}{log\PYZus{}data}\PY{p}{[}\PY{n}{feature}\PY{p}{]} \PY{o}{\PYZlt{}}\PY{o}{=} \PY{n}{Q3} \PY{o}{+} \PY{n}{step}\PY{p}{)}\PY{p}{)}\PY{p}{]}\PY{o}{.}\PY{n}{index}\PY{o}{.}\PY{n}{tolist}\PY{p}{(}\PY{p}{)}\PY{p}{:}
                 \PY{k}{if} \PY{n}{entry} \PY{o+ow}{in} \PY{n}{outlier\PYZus{}indices}\PY{p}{:}
                     \PY{n}{outlier\PYZus{}indices}\PY{p}{[}\PY{n}{entry}\PY{p}{]} \PY{o}{+}\PY{o}{=} \PY{l+m+mi}{1}
                 \PY{k}{else}\PY{p}{:}
                     \PY{n}{outlier\PYZus{}indices}\PY{p}{[}\PY{n}{entry}\PY{p}{]} \PY{o}{=} \PY{l+m+mi}{1}
         
         \PY{c+c1}{\PYZsh{} OPTIONAL: Select the indices for data points you wish to remove}
         \PY{n}{minFeaturesForOutlier} \PY{o}{=} \PY{l+m+mi}{2}
         \PY{n}{outliers}  \PY{o}{=} \PY{p}{[}\PY{p}{]}
         
         \PY{k}{for} \PY{n}{key} \PY{o+ow}{in} \PY{n}{outlier\PYZus{}indices}\PY{p}{:}
             \PY{k}{if}\PY{p}{(}\PY{n}{outlier\PYZus{}indices}\PY{p}{[}\PY{n}{key}\PY{p}{]} \PY{o}{\PYZgt{}}\PY{o}{=} \PY{n}{minFeaturesForOutlier}\PY{p}{)}\PY{p}{:}
                 \PY{n}{outliers}\PY{o}{.}\PY{n}{append}\PY{p}{(}\PY{n}{key}\PY{p}{)}
         
         \PY{k}{print} \PY{l+s+s2}{\PYZdq{}}\PY{l+s+s2}{The samples that were removed on qualifying minFeaturesForOutlier of \PYZob{}\PYZcb{} are:}\PY{l+s+s2}{\PYZdq{}}\PY{o}{.}\PY{n}{format}\PY{p}{(}\PY{n}{minFeaturesForOutlier}\PY{p}{)}
         \PY{n}{display}\PY{p}{(}\PY{n}{log\PYZus{}data}\PY{o}{.}\PY{n}{loc}\PY{p}{[}\PY{n}{outliers}\PY{p}{]}\PY{p}{)}
         
         \PY{c+c1}{\PYZsh{} Remove the outliers, if any were specified}
         \PY{n}{good\PYZus{}data} \PY{o}{=} \PY{n}{log\PYZus{}data}\PY{o}{.}\PY{n}{drop}\PY{p}{(}\PY{n}{log\PYZus{}data}\PY{o}{.}\PY{n}{index}\PY{p}{[}\PY{n}{outliers}\PY{p}{]}\PY{p}{)}\PY{o}{.}\PY{n}{reset\PYZus{}index}\PY{p}{(}\PY{n}{drop} \PY{o}{=} \PY{n+nb+bp}{True}\PY{p}{)}
\end{Verbatim}

    \begin{Verbatim}[commandchars=\\\{\}]
Description of the log transformed dataset:
    \end{Verbatim}

    
    \begin{verbatim}
            Fresh        Milk     Grocery      Frozen  Detergents_Paper  \
count  440.000000  440.000000  440.000000  440.000000        440.000000   
mean     8.730544    8.121047    8.441169    7.301396          6.785972   
std      1.480071    1.081365    1.116172    1.284540          1.721020   
min      1.098612    4.007333    1.098612    3.218876          1.098612   
25%      8.048059    7.334981    7.674616    6.609678          5.548101   
50%      9.048286    8.196159    8.467057    7.330388          6.705018   
75%      9.737064    8.880480    9.273854    8.175896          8.274341   
max     11.627601   11.205013   11.437986   11.016479         10.617099   

       Delicatessen  
count    440.000000  
mean       6.665133  
std        1.310832  
min        1.098612  
25%        6.011875  
50%        6.872645  
75%        7.506728  
max       10.777768  
    \end{verbatim}

    
    \begin{Verbatim}[commandchars=\\\{\}]
Data points considered outliers for the feature 'Fresh':
    \end{Verbatim}

    
    \begin{verbatim}
        Fresh       Milk    Grocery    Frozen  Detergents_Paper  Delicatessen
65   4.442651   9.950323  10.732651  3.583519         10.095388      7.260523
66   2.197225   7.335634   8.911530  5.164786          8.151333      3.295837
81   5.389072   9.163249   9.575192  5.645447          8.964184      5.049856
95   1.098612   7.979339   8.740657  6.086775          5.407172      6.563856
96   3.135494   7.869402   9.001839  4.976734          8.262043      5.379897
128  4.941642   9.087834   8.248791  4.955827          6.967909      1.098612
171  5.298317  10.160530   9.894245  6.478510          9.079434      8.740337
193  5.192957   8.156223   9.917982  6.865891          8.633731      6.501290
218  2.890372   8.923191   9.629380  7.158514          8.475746      8.759669
304  5.081404   8.917311  10.117510  6.424869          9.374413      7.787382
305  5.493061   9.468001   9.088399  6.683361          8.271037      5.351858
338  1.098612   5.808142   8.856661  9.655090          2.708050      6.309918
353  4.762174   8.742574   9.961898  5.429346          9.069007      7.013016
355  5.247024   6.588926   7.606885  5.501258          5.214936      4.844187
357  3.610918   7.150701  10.011086  4.919981          8.816853      4.700480
412  4.574711   8.190077   9.425452  4.584967          7.996317      4.127134
    \end{verbatim}

    
    \begin{Verbatim}[commandchars=\\\{\}]
Data points considered outliers for the feature 'Milk':
    \end{Verbatim}

    
    \begin{verbatim}
         Fresh       Milk    Grocery    Frozen  Detergents_Paper  Delicatessen
86   10.039983  11.205013  10.377047  6.894670          9.906981      6.805723
98    6.220590   4.718499   6.656727  6.796824          4.025352      4.882802
154   6.432940   4.007333   4.919981  4.317488          1.945910      2.079442
356  10.029503   4.897840   5.384495  8.057377          2.197225      6.306275
    \end{verbatim}

    
    \begin{Verbatim}[commandchars=\\\{\}]
Data points considered outliers for the feature 'Grocery':
    \end{Verbatim}

    
    \begin{verbatim}
        Fresh      Milk   Grocery    Frozen  Detergents_Paper  Delicatessen
75   9.923192  7.036148  1.098612  8.390949          1.098612      6.882437
154  6.432940  4.007333  4.919981  4.317488          1.945910      2.079442
    \end{verbatim}

    
    \begin{Verbatim}[commandchars=\\\{\}]
Data points considered outliers for the feature 'Frozen':
    \end{Verbatim}

    
    \begin{verbatim}
         Fresh      Milk    Grocery     Frozen  Detergents_Paper  Delicatessen
38    8.431853  9.663261   9.723703   3.496508          8.847360      6.070738
57    8.597297  9.203618   9.257892   3.637586          8.932213      7.156177
65    4.442651  9.950323  10.732651   3.583519         10.095388      7.260523
145  10.000569  9.034080  10.457143   3.737670          9.440738      8.396155
175   7.759187  8.967632   9.382106   3.951244          8.341887      7.436617
264   6.978214  9.177714   9.645041   4.110874          8.696176      7.142827
325  10.395650  9.728181   9.519735  11.016479          7.148346      8.632128
420   8.402007  8.569026   9.490015   3.218876          8.827321      7.239215
429   9.060331  7.467371   8.183118   3.850148          4.430817      7.824446
439   7.932721  7.437206   7.828038   4.174387          6.167516      3.951244
    \end{verbatim}

    
    \begin{Verbatim}[commandchars=\\\{\}]
Data points considered outliers for the feature 'Detergents\_Paper':
    \end{Verbatim}

    
    \begin{verbatim}
        Fresh      Milk   Grocery    Frozen  Detergents_Paper  Delicatessen
75   9.923192  7.036148  1.098612  8.390949          1.098612      6.882437
161  9.428190  6.291569  5.645447  6.995766          1.098612      7.711101
    \end{verbatim}

    
    \begin{Verbatim}[commandchars=\\\{\}]
Data points considered outliers for the feature 'Delicatessen':
    \end{Verbatim}

    
    \begin{verbatim}
         Fresh       Milk    Grocery     Frozen  Detergents_Paper  \
66    2.197225   7.335634   8.911530   5.164786          8.151333   
109   7.248504   9.724899  10.274568   6.511745          6.728629   
128   4.941642   9.087834   8.248791   4.955827          6.967909   
137   8.034955   8.997147   9.021840   6.493754          6.580639   
142  10.519646   8.875147   9.018332   8.004700          2.995732   
154   6.432940   4.007333   4.919981   4.317488          1.945910   
183  10.514529  10.690808   9.911952  10.505999          5.476464   
184   5.789960   6.822197   8.457443   4.304065          5.811141   
187   7.798933   8.987447   9.192075   8.743372          8.148735   
203   6.368187   6.529419   7.703459   6.150603          6.860664   
233   6.871091   8.513988   8.106515   6.842683          6.013715   
285  10.602965   6.461468   8.188689   6.948897          6.077642   
289  10.663966   5.655992   6.154858   7.235619          3.465736   
343   7.431892   8.848509  10.177932   7.283448          9.646593   

     Delicatessen  
66       3.295837  
109      1.098612  
128      1.098612  
137      3.583519  
142      1.098612  
154      2.079442  
183     10.777768  
184      2.397895  
187      1.098612  
203      2.890372  
233      1.945910  
285      2.890372  
289      3.091042  
343      3.610918  
    \end{verbatim}

    
    \begin{Verbatim}[commandchars=\\\{\}]
The samples that were removed on qualifying minFeaturesForOutlier of 2 are:
    \end{Verbatim}

    
    \begin{verbatim}
        Fresh      Milk    Grocery    Frozen  Detergents_Paper  Delicatessen
128  4.941642  9.087834   8.248791  4.955827          6.967909      1.098612
154  6.432940  4.007333   4.919981  4.317488          1.945910      2.079442
65   4.442651  9.950323  10.732651  3.583519         10.095388      7.260523
66   2.197225  7.335634   8.911530  5.164786          8.151333      3.295837
75   9.923192  7.036148   1.098612  8.390949          1.098612      6.882437
    \end{verbatim}

    
    \subsubsection{Question 4}\label{question-4}

\emph{Are there any data points considered outliers for more than one
feature? Should these data points be removed from the dataset? If any
data points were added to the \texttt{outliers} list to be removed,
explain why.}

    \textbf{Answer:}

There are a couple of data points that have outliers for more than one
feature viz., 65, 66, 75, 128 and 154.

Since Tukey's method of outlier detection is only based on cutting off
the tails of a distribution --- the cut-off point being rather
generalized to all kinds of distribution ---, it can't objectively be
generalized to all cases. In reality, an outlier and its relative
importance in the dataset could be influenced by many factors. For
instance, consider datapoint 75; it has been categorized as outlier for
the features `Grocery' and `Detergents\_Paper'. It is important to note
that both the feature values lie in the lower cut-off regions.
Considering that the other feature values (or consumption in that
category) are around 75\% of their distributions, it may not really
qualify for a real-world business. That is, no known type of business
could be buying so many goods of all types except groceries and
detergents which eerily look like the least consumed. On the contrary,
the business in reality may be a very low consumer of groceries and
detergents, and so the data point could be completely valid.
Alternatively, another business, like data point 65 may be considered
outlier for being on the upper cut-off. That is, some goods are consumed
much more than the average. But, the decision not to remove this data
point may be motivated by the fact that the representation of very big
businesses in the dataset is required at all costs. Only an agent
(person or machine) well acquainted with the art or possessing
considerable prior knowledge about the data sources would be able to
determine the applicabilty of outlier better, but still may not be
objective.

For the purposes of this project, an easy scheme of detecting outliers
is chosen. If any data point has more than one feature
(\textgreater{}=2) considered an outlier based on Tukey's method, that
data point is considered an outlier and removed.

    \subsection{Feature Transformation}\label{feature-transformation}

In this section you will use principal component analysis (PCA) to draw
conclusions about the underlying structure of the wholesale customer
data. Since using PCA on a dataset calculates the dimensions which best
maximize variance, we will find which compound combinations of features
best describe customers.

    \subsubsection{Implementation: PCA}\label{implementation-pca}

Now that the data has been scaled to a more normal distribution and has
had any necessary outliers removed, we can now apply PCA to the
\texttt{good\_data} to discover which dimensions about the data best
maximize the variance of features involved. In addition to finding these
dimensions, PCA will also report the \emph{explained variance ratio} of
each dimension --- how much variance within the data is explained by
that dimension alone.

In the code block below, you will need to implement the following: -
Import \texttt{sklearn.preprocessing.PCA} and assign the results of
fitting PCA in six dimensions with \texttt{good\_data} to \texttt{pca}.
- Apply a PCA transformation of the sample log-data
\texttt{log\_samples} using \texttt{pca.transform}, and assign the
results to \texttt{pca\_samples}.

    \begin{Verbatim}[commandchars=\\\{\}]
{\color{incolor}In [{\color{incolor}11}]:} \PY{c+c1}{\PYZsh{} Apply PCA to the good data with the same number of dimensions as features}
         \PY{k+kn}{from} \PY{n+nn}{sklearn.decomposition} \PY{k+kn}{import} \PY{n}{PCA}
         \PY{n}{pca} \PY{o}{=} \PY{n}{PCA}\PY{p}{(}\PY{n}{n\PYZus{}components}\PY{o}{=}\PY{n+nb}{len}\PY{p}{(}\PY{n}{good\PYZus{}data}\PY{o}{.}\PY{n}{keys}\PY{p}{(}\PY{p}{)}\PY{p}{)}\PY{p}{)}
         \PY{n}{pca}\PY{o}{.}\PY{n}{fit}\PY{p}{(}\PY{n}{good\PYZus{}data}\PY{p}{)}
         
         \PY{c+c1}{\PYZsh{} Apply a PCA transformation to the sample log\PYZhy{}data}
         \PY{n}{pca\PYZus{}samples} \PY{o}{=} \PY{n}{pca}\PY{o}{.}\PY{n}{transform}\PY{p}{(}\PY{n}{log\PYZus{}samples}\PY{p}{)}
         
         \PY{c+c1}{\PYZsh{} Generate PCA results plot}
         \PY{n}{pca\PYZus{}results} \PY{o}{=} \PY{n}{rs}\PY{o}{.}\PY{n}{pca\PYZus{}results}\PY{p}{(}\PY{n}{good\PYZus{}data}\PY{p}{,} \PY{n}{pca}\PY{p}{)}
         
         \PY{c+c1}{\PYZsh{} Print explained variances}
         \PY{k}{print} \PY{l+s+s2}{\PYZdq{}}\PY{l+s+s2}{Explained ratio of first principal component is \PYZob{}\PYZcb{}}\PY{l+s+s2}{\PYZdq{}}\PY{o}{.}\PY{n}{format}\PY{p}{(}\PY{n}{pca}\PY{o}{.}\PY{n}{explained\PYZus{}variance\PYZus{}ratio\PYZus{}}\PY{p}{[}\PY{l+m+mi}{0}\PY{p}{]}\PY{p}{)}
         \PY{k}{print} \PY{l+s+s2}{\PYZdq{}}\PY{l+s+s2}{Explained ratio of first and second principal component is \PYZob{}\PYZcb{}}\PY{l+s+s2}{\PYZdq{}}\PY{o}{.}\PY{n}{format}\PY{p}{(}\PY{n}{np}\PY{o}{.}\PY{n}{sum}\PY{p}{(}\PY{n}{pca}\PY{o}{.}\PY{n}{explained\PYZus{}variance\PYZus{}ratio\PYZus{}}\PY{p}{[}\PY{p}{:}\PY{l+m+mi}{2}\PY{p}{]}\PY{p}{)}\PY{p}{)}
         \PY{k}{print} \PY{l+s+s2}{\PYZdq{}}\PY{l+s+s2}{Explained ratio of first four principal components is \PYZob{}\PYZcb{}}\PY{l+s+s2}{\PYZdq{}}\PY{o}{.}\PY{n}{format}\PY{p}{(}\PY{n}{np}\PY{o}{.}\PY{n}{sum}\PY{p}{(}\PY{n}{pca}\PY{o}{.}\PY{n}{explained\PYZus{}variance\PYZus{}ratio\PYZus{}}\PY{p}{[}\PY{p}{:}\PY{l+m+mi}{4}\PY{p}{]}\PY{p}{)}\PY{p}{)}
\end{Verbatim}

    \begin{Verbatim}[commandchars=\\\{\}]
Explained ratio of first principal component is 0.44302504749
Explained ratio of first and second principal component is 0.706817230807
Explained ratio of first four principal components is 0.931090109951
    \end{Verbatim}

    \begin{center}
    \adjustimage{max size={0.9\linewidth}{0.9\paperheight}}{customer_segments_files/customer_segments_32_1.png}
    \end{center}
    { \hspace*{\fill} \\}
    
    \subsubsection{Question 5}\label{question-5}

\emph{How much variance in the data is explained} \textbf{\emph{in
total}} \emph{by the first and second principal component? What about
the first four principal components? Using the visualization provided
above, discuss what the first four dimensions best represent in terms of
customer spending.}\\
\textbf{Hint:} A positive increase in a specific dimension corresponds
with an \emph{increase} of the \emph{positive-weighted} features and a
\emph{decrease} of the \emph{negative-weighted} features. The rate of
increase or decrease is based on the indivdual feature weights.

    \textbf{Answer:}

The first two principal components explain \texttt{70.68\%} of total
variance, and the first four principal components explain
\texttt{93.1\%} total variance.

For ease, the following interpretation of each principal component only
considers those features that have weights above magnitude
\textasciitilde{}0.4:

\begin{longtable}[c]{@{}cc@{}}
\toprule
\begin{minipage}[b]{0.26\columnwidth}\centering\strut
Principal component
\strut\end{minipage} &
\begin{minipage}[b]{0.57\columnwidth}\centering\strut
Reinterpreted in terms of customer spending
\strut\end{minipage}\tabularnewline
\midrule
\endhead
\begin{minipage}[t]{0.26\columnwidth}\centering\strut
First
\strut\end{minipage} &
\begin{minipage}[t]{0.57\columnwidth}\centering\strut
This component increases with increase in `Milk', `Grocery' and
`Detergents\_Paper' with highest emphasis on the last category. In
effect, this component can be reinterpreted as a measure of spending on
milk, grocery and detergents products, with detergents products having
the highest influence. Also, the component reveals that all three
product categories are highly correlated with each other. That is,
generally, increase in one category follows increase in other categories
too.
\strut\end{minipage}\tabularnewline
\begin{minipage}[t]{0.26\columnwidth}\centering\strut
Second
\strut\end{minipage} &
\begin{minipage}[t]{0.57\columnwidth}\centering\strut
This component increases with increase in `Fresh', `Frozen' and
`Delicatessen', with most emphasis on the last category. So this
component can serve as a measure of spending on frozen, fresh and fine
food products. Furthermore, spending on fresh, frozen and fine food
products generally follow the same trend.
\strut\end{minipage}\tabularnewline
\begin{minipage}[t]{0.26\columnwidth}\centering\strut
Third
\strut\end{minipage} &
\begin{minipage}[t]{0.57\columnwidth}\centering\strut
This component increases with increase in `Delicatessen' and decrease in
`Fresh', and could represent a measure of heavy consumption (spending)
of fine food products \textbf{with} very low consumption of fresh
products.
\strut\end{minipage}\tabularnewline
\begin{minipage}[t]{0.26\columnwidth}\centering\strut
Fourth
\strut\end{minipage} &
\begin{minipage}[t]{0.57\columnwidth}\centering\strut
This component increases with increase in `Frozen' and decrease in
`Delicatessen'. It could represent a measure of combined high
consumption of frozen products \textbf{and} low consumption of fine food
products.
\strut\end{minipage}\tabularnewline
\bottomrule
\end{longtable}

    \subsubsection{Observation}\label{observation}

Run the code below to see how the log-transformed sample data has
changed after having a PCA transformation applied to it in six
dimensions. Observe the numerical value for the first four dimensions of
the sample points. Consider if this is consistent with your initial
interpretation of the sample points.

    \begin{Verbatim}[commandchars=\\\{\}]
{\color{incolor}In [{\color{incolor}12}]:} \PY{c+c1}{\PYZsh{} Display sample log\PYZhy{}data after having a PCA transformation applied}
         \PY{n}{display}\PY{p}{(}\PY{n}{pd}\PY{o}{.}\PY{n}{DataFrame}\PY{p}{(}\PY{n}{np}\PY{o}{.}\PY{n}{round}\PY{p}{(}\PY{n}{pca\PYZus{}samples}\PY{p}{,} \PY{l+m+mi}{4}\PY{p}{)}\PY{p}{,} \PY{n}{columns} \PY{o}{=} \PY{n}{pca\PYZus{}results}\PY{o}{.}\PY{n}{index}\PY{o}{.}\PY{n}{values}\PY{p}{)}\PY{p}{)}
\end{Verbatim}

    
    \begin{verbatim}
   Dimension 1  Dimension 2  Dimension 3  Dimension 4  Dimension 5  \
0       2.3579       1.7393       0.2210       0.2840       0.5939   
1      -0.6058       0.6931      -0.1705       0.5478      -0.0380   
2       2.3804      -2.8989       0.9016       1.2539      -0.6757   

   Dimension 6  
0       0.0148  
1      -0.6545  
2       0.6118  
    \end{verbatim}

    
    \subsubsection{Implementation: Dimensionality
Reduction}\label{implementation-dimensionality-reduction}

When using principal component analysis, one of the main goals is to
reduce the dimensionality of the data --- in effect, reducing the
complexity of the problem. Dimensionality reduction comes at a cost:
Fewer dimensions used implies less of the total variance in the data is
being explained. Because of this, the \emph{cumulative explained
variance ratio} is extremely important for knowing how many dimensions
are necessary for the problem. Additionally, if a signifiant amount of
variance is explained by only two or three dimensions, the reduced data
can be visualized afterwards.

In the code block below, you will need to implement the following: -
Assign the results of fitting PCA in two dimensions with
\texttt{good\_data} to \texttt{pca}. - Apply a PCA transformation of
\texttt{good\_data} using \texttt{pca.transform}, and assign the reuslts
to \texttt{reduced\_data}. - Apply a PCA transformation of the sample
log-data \texttt{log\_samples} using \texttt{pca.transform}, and assign
the results to \texttt{pca\_samples}.

    \begin{Verbatim}[commandchars=\\\{\}]
{\color{incolor}In [{\color{incolor}13}]:} \PY{c+c1}{\PYZsh{} Fit PCA to the good data using only two dimensions}
         \PY{n}{pca} \PY{o}{=} \PY{n}{PCA}\PY{p}{(}\PY{n}{n\PYZus{}components}\PY{o}{=}\PY{l+m+mi}{2}\PY{p}{)}
         \PY{n}{pca}\PY{o}{.}\PY{n}{fit}\PY{p}{(}\PY{n}{good\PYZus{}data}\PY{p}{)}
         
         \PY{c+c1}{\PYZsh{} Apply a PCA transformation the good data}
         \PY{n}{reduced\PYZus{}data} \PY{o}{=} \PY{n}{pca}\PY{o}{.}\PY{n}{transform}\PY{p}{(}\PY{n}{good\PYZus{}data}\PY{p}{)}
         
         \PY{c+c1}{\PYZsh{} Apply a PCA transformation to the sample log\PYZhy{}data}
         \PY{n}{pca\PYZus{}samples} \PY{o}{=} \PY{n}{pca}\PY{o}{.}\PY{n}{transform}\PY{p}{(}\PY{n}{log\PYZus{}samples}\PY{p}{)}
         
         \PY{c+c1}{\PYZsh{} Create a DataFrame for the reduced data}
         \PY{n}{reduced\PYZus{}data} \PY{o}{=} \PY{n}{pd}\PY{o}{.}\PY{n}{DataFrame}\PY{p}{(}\PY{n}{reduced\PYZus{}data}\PY{p}{,} \PY{n}{columns} \PY{o}{=} \PY{p}{[}\PY{l+s+s1}{\PYZsq{}}\PY{l+s+s1}{Dimension 1}\PY{l+s+s1}{\PYZsq{}}\PY{p}{,} \PY{l+s+s1}{\PYZsq{}}\PY{l+s+s1}{Dimension 2}\PY{l+s+s1}{\PYZsq{}}\PY{p}{]}\PY{p}{)}
\end{Verbatim}

    \paragraph{Additional:}\label{additional}

    \begin{Verbatim}[commandchars=\\\{\}]
{\color{incolor}In [{\color{incolor}14}]:} \PY{c+c1}{\PYZsh{} Make a joint grid of the first two dimensions\PYZsq{} KDE}
         \PY{n}{g} \PY{o}{=} \PY{n}{sns}\PY{o}{.}\PY{n}{JointGrid}\PY{p}{(}\PY{l+s+s2}{\PYZdq{}}\PY{l+s+s2}{Dimension 1}\PY{l+s+s2}{\PYZdq{}}\PY{p}{,} \PY{l+s+s2}{\PYZdq{}}\PY{l+s+s2}{Dimension 2}\PY{l+s+s2}{\PYZdq{}}\PY{p}{,} \PY{n}{reduced\PYZus{}data}\PY{p}{,} \PY{n}{xlim}\PY{o}{=}\PY{p}{(}\PY{o}{\PYZhy{}}\PY{l+m+mi}{6}\PY{p}{,}\PY{l+m+mi}{6}\PY{p}{)}\PY{p}{,} \PY{n}{ylim}\PY{o}{=}\PY{p}{(}\PY{o}{\PYZhy{}}\PY{l+m+mi}{5}\PY{p}{,}\PY{l+m+mi}{5}\PY{p}{)}\PY{p}{)}
         \PY{n}{g} \PY{o}{=} \PY{n}{g}\PY{o}{.}\PY{n}{plot\PYZus{}joint}\PY{p}{(}\PY{n}{sns}\PY{o}{.}\PY{n}{kdeplot}\PY{p}{,} \PY{n}{cmap}\PY{o}{=}\PY{l+s+s2}{\PYZdq{}}\PY{l+s+s2}{Blues}\PY{l+s+s2}{\PYZdq{}}\PY{p}{,} \PY{n}{shade}\PY{o}{=}\PY{n+nb+bp}{True}\PY{p}{)}
         \PY{n}{g} \PY{o}{=} \PY{n}{g}\PY{o}{.}\PY{n}{plot\PYZus{}marginals}\PY{p}{(}\PY{n}{sns}\PY{o}{.}\PY{n}{kdeplot}\PY{p}{,} \PY{n}{shade}\PY{o}{=}\PY{n+nb+bp}{True}\PY{p}{)}
\end{Verbatim}

    \begin{center}
    \adjustimage{max size={0.9\linewidth}{0.9\paperheight}}{customer_segments_files/customer_segments_40_0.png}
    \end{center}
    { \hspace*{\fill} \\}
    
    \subsubsection{Observation}\label{observation}

Run the code below to see how the log-transformed sample data has
changed after having a PCA transformation applied to it using only two
dimensions. Observe how the values for the first two dimensions remains
unchanged when compared to a PCA transformation in six dimensions.

    \begin{Verbatim}[commandchars=\\\{\}]
{\color{incolor}In [{\color{incolor}15}]:} \PY{c+c1}{\PYZsh{} Display sample log\PYZhy{}data after applying PCA transformation in two dimensions}
         \PY{n}{display}\PY{p}{(}\PY{n}{pd}\PY{o}{.}\PY{n}{DataFrame}\PY{p}{(}\PY{n}{np}\PY{o}{.}\PY{n}{round}\PY{p}{(}\PY{n}{pca\PYZus{}samples}\PY{p}{,} \PY{l+m+mi}{4}\PY{p}{)}\PY{p}{,} \PY{n}{columns} \PY{o}{=} \PY{p}{[}\PY{l+s+s1}{\PYZsq{}}\PY{l+s+s1}{Dimension 1}\PY{l+s+s1}{\PYZsq{}}\PY{p}{,} \PY{l+s+s1}{\PYZsq{}}\PY{l+s+s1}{Dimension 2}\PY{l+s+s1}{\PYZsq{}}\PY{p}{]}\PY{p}{)}\PY{p}{)}
\end{Verbatim}

    
    \begin{verbatim}
   Dimension 1  Dimension 2
0       2.3579       1.7393
1      -0.6058       0.6931
2       2.3804      -2.8989
    \end{verbatim}

    
    \subsection{Clustering}\label{clustering}

In this section, you will choose to use either a K-Means clustering
algorithm or a Gaussian Mixture Model clustering algorithm to identify
the various customer segments hidden in the data. You will then recover
specific data points from the clusters to understand their significance
by transforming them back into their original dimension and scale.

    \subsubsection{Question 6}\label{question-6}

\emph{What are the advantages to using a K-Means clustering algorithm?
What are the advantages to using a Gaussian Mixture Model clustering
algorithm? Given your observations about the wholesale customer data so
far, which of the two algorithms will you use and why?}

    \textbf{Answer:}

\href{http://scikit-learn.org/stable/modules/clustering.html\#k-means}{K-Means}
is one of the simplest, yet powerful form of clustering algorithm, in
which clusters are found by minimizing within-cluster sum-of-squares.
Advantages to using K-Means in general are: - It scales very well to
large datasets - It is fast and easily parallelizable - It can serve as
a good general purpose clustering algorithm

\href{http://scikit-learn.org/stable/modules/mixture.html\#gmm-classifier}{GMM}
on the other hand is a more generalized clustering algorithm and usually
relies on expectation-maximization. Advantages to using GMM in general
are: - It generalizes the K-Means algorithm - It can cluster data with
dissimlar variances/shapes and sizes - It can directly be used to gain
insight into data beyond clustering such as density estimation - It can
effectively facilitate a confidence score for each data point

Both K-Means and GMM, roughly falling in the same category of clustering
methods, have the following disadvantages: - Number of clusters in the
data should be known a priori, although metrics such as silhoutte score
or Bayesian information criterion can be used to find the most likely
number of clusers - It is essential the data posseses a flat geometry -
Local optima are unavoidable, and to counter this, both make use of
various passes with random starting points

Given any dataset, the first thing to try would be K-Means as it is
known to fit well for various application domains. For the wholesale
customer dataset too, K-Means naturally fits in as the first try. But,
from the scatter matrix plots above, we know that the all features (log
tranformed) have roughly Gaussian distributions with more or less
similar standard deviations, and each pair of features (except the
highly correlated ones) show similar shapes, making it suitable for
K-Means. Furthermore, considering that PCA is a linear transformation,
and the correlated features have effectively very less influence in the
first two principal components, the conditions should be ideal for
K-Means, with convex, isotropic data points. Lastly, although it doesn't
make a practical difference for the small dataset used in this project,
it is known that K-Means scales well and is very efficient
computationally in comparison to GMM (preliminary tests with the data
showed K-Means performing roughly 3x faster than GMM). So in case more
data is available, K-Means should still be the best contender on the PCA
transformed dataset.

Note that GMM based clustering could be used in the project, but it
would most likely be an overkill.

    \subsubsection{Implementation: Creating
Clusters}\label{implementation-creating-clusters}

Depending on the problem, the number of clusters that you expect to be
in the data may already be known. When the number of clusters is not
known \emph{a priori}, there is no guarantee that a given number of
clusters best segments the data, since it is unclear what structure
exists in the data --- if any. However, we can quantify the ``goodness''
of a clustering by calculating each data point's \emph{silhouette
coefficient}. The
\href{http://scikit-learn.org/stable/modules/generated/sklearn.metrics.silhouette_score.html}{silhouette
coefficient} for a data point measures how similar it is to its assigned
cluster from -1 (dissimilar) to 1 (similar). Calculating the \emph{mean}
silhouette coefficient provides for a simple scoring method of a given
clustering.

In the code block below, you will need to implement the following: - Fit
a clustering algorithm to the \texttt{reduced\_data} and assign it to
\texttt{clusterer}. - Predict the cluster for each data point in
\texttt{reduced\_data} using \texttt{clusterer.predict} and assign them
to \texttt{preds}. - Find the cluster centers using the algorithm's
respective attribute and assign them to \texttt{centers}. - Predict the
cluster for each sample data point in \texttt{pca\_samples} and assign
them \texttt{sample\_preds}. - Import sklearn.metrics.silhouette\_score
and calculate the silhouette score of \texttt{reduced\_data} against
\texttt{preds}. - Assign the silhouette score to \texttt{score} and
print the result.

    \begin{Verbatim}[commandchars=\\\{\}]
{\color{incolor}In [{\color{incolor}16}]:} \PY{k+kn}{from} \PY{n+nn}{sklearn.cluster} \PY{k+kn}{import} \PY{n}{KMeans}
         \PY{k+kn}{from} \PY{n+nn}{sklearn.metrics} \PY{k+kn}{import} \PY{n}{silhouette\PYZus{}score}
         
         \PY{c+c1}{\PYZsh{} Number of clusters to experiment on}
         \PY{n}{num\PYZus{}clusters\PYZus{}list} \PY{o}{=} \PY{n+nb}{range}\PY{p}{(}\PY{l+m+mi}{2}\PY{p}{,}\PY{l+m+mi}{11}\PY{p}{)}
         
         \PY{c+c1}{\PYZsh{} Chosen (best) number of clusters post\PYZhy{}experimentation}
         \PY{n}{desired\PYZus{}num\PYZus{}clusters} \PY{o}{=} \PY{l+m+mi}{2}
         
         \PY{c+c1}{\PYZsh{} Append chosen number to the last of the search list so that the last results are retained}
         \PY{n}{num\PYZus{}clusters\PYZus{}list}\PY{o}{.}\PY{n}{append}\PY{p}{(}\PY{n}{desired\PYZus{}num\PYZus{}clusters}\PY{p}{)}
         
         \PY{c+c1}{\PYZsh{} Try with various number of clusters}
         \PY{k}{for} \PY{n}{num\PYZus{}clusters} \PY{o+ow}{in} \PY{n}{num\PYZus{}clusters\PYZus{}list}\PY{p}{:}
             \PY{c+c1}{\PYZsh{} Apply your clustering algorithm of choice to the reduced data }
             \PY{n}{clusterer} \PY{o}{=} \PY{n}{KMeans}\PY{p}{(}\PY{n}{n\PYZus{}clusters}\PY{o}{=}\PY{n}{num\PYZus{}clusters}\PY{p}{,} \PY{n}{random\PYZus{}state}\PY{o}{=}\PY{n}{random\PYZus{}state}\PY{p}{,} \PY{n}{n\PYZus{}init}\PY{o}{=}\PY{l+m+mi}{8}\PY{p}{,} \PY{n}{n\PYZus{}jobs}\PY{o}{=}\PY{l+m+mi}{4}\PY{p}{)}
             \PY{n}{clusterer}\PY{o}{.}\PY{n}{fit}\PY{p}{(}\PY{n}{reduced\PYZus{}data}\PY{p}{)}
         
             \PY{c+c1}{\PYZsh{} Predict the cluster for each data point}
             \PY{n}{preds} \PY{o}{=} \PY{n}{clusterer}\PY{o}{.}\PY{n}{predict}\PY{p}{(}\PY{n}{reduced\PYZus{}data}\PY{p}{)}
         
             \PY{c+c1}{\PYZsh{} Find the cluster centers}
             \PY{n}{centers} \PY{o}{=} \PY{n}{clusterer}\PY{o}{.}\PY{n}{cluster\PYZus{}centers\PYZus{}}
         
             \PY{c+c1}{\PYZsh{} Predict the cluster for each transformed sample data point}
             \PY{n}{sample\PYZus{}preds} \PY{o}{=} \PY{n}{clusterer}\PY{o}{.}\PY{n}{predict}\PY{p}{(}\PY{n}{pca\PYZus{}samples}\PY{p}{)}
         
             \PY{c+c1}{\PYZsh{} Calculate the mean silhouette coefficient for the number of clusters chosen}
             \PY{n}{score} \PY{o}{=} \PY{n}{silhouette\PYZus{}score}\PY{p}{(}\PY{n}{reduced\PYZus{}data}\PY{p}{,} \PY{n}{preds}\PY{p}{)}
             
             \PY{c+c1}{\PYZsh{} Print mean silhoutte score for each number of chosen clusters}
             \PY{k}{print} \PY{l+s+s2}{\PYZdq{}}\PY{l+s+s2}{Mean silhouette score for \PYZob{}\PYZcb{} clusters is \PYZob{}:.4f\PYZcb{}}\PY{l+s+s2}{\PYZdq{}}\PY{o}{.}\PY{n}{format}\PY{p}{(}\PY{n}{num\PYZus{}clusters}\PY{p}{,} \PY{n}{score}\PY{p}{)}
\end{Verbatim}

    \begin{Verbatim}[commandchars=\\\{\}]
Mean silhouette score for 2 clusters is 0.4263
Mean silhouette score for 3 clusters is 0.3974
Mean silhouette score for 4 clusters is 0.3315
Mean silhouette score for 5 clusters is 0.3500
Mean silhouette score for 6 clusters is 0.3619
Mean silhouette score for 7 clusters is 0.3619
Mean silhouette score for 8 clusters is 0.3491
Mean silhouette score for 9 clusters is 0.3633
Mean silhouette score for 10 clusters is 0.3520
Mean silhouette score for 2 clusters is 0.4263
    \end{Verbatim}

    \subsubsection{Question 7}\label{question-7}

\emph{Report the silhouette score for several cluster numbers you tried.
Of these, which number of clusters has the best silhouette score?}

    \textbf{Answer:}

Mean silhoutte scores for 2 clusters through to 10 clusters were
calculated above. Silhoutte score for 2 clusters is 0.4263 and that for
3 clusters is 0.3974. For higher number of clusters, the silhoutte score
saturates (or most likely follows a very shallow dip) near
\textasciitilde{}0.34.

The best silhouette score is for 2 clusters and the same will be used in
the further sections.

    \subsubsection{Cluster Visualization}\label{cluster-visualization}

Once you've chosen the optimal number of clusters for your clustering
algorithm using the scoring metric above, you can now visualize the
results by executing the code block below. Note that, for
experimentation purposes, you are welcome to adjust the number of
clusters for your clustering algorithm to see various visualizations.
The final visualization provided should, however, correspond with the
optimal number of clusters.

    \begin{Verbatim}[commandchars=\\\{\}]
{\color{incolor}In [{\color{incolor}17}]:} \PY{c+c1}{\PYZsh{} Display the results of the clustering from implementation}
         \PY{n}{rs}\PY{o}{.}\PY{n}{cluster\PYZus{}results}\PY{p}{(}\PY{n}{reduced\PYZus{}data}\PY{p}{,} \PY{n}{preds}\PY{p}{,} \PY{n}{centers}\PY{p}{,} \PY{n}{pca\PYZus{}samples}\PY{p}{)}
\end{Verbatim}

    \begin{center}
    \adjustimage{max size={0.9\linewidth}{0.9\paperheight}}{customer_segments_files/customer_segments_51_0.png}
    \end{center}
    { \hspace*{\fill} \\}
    
    \subsubsection{Implementation: Data
Recovery}\label{implementation-data-recovery}

Each cluster present in the visualization above has a central point.
These centers (or means) are not specifically data points from the data,
but rather the \emph{averages} of all the data points predicted in the
respective clusters. For the problem of creating customer segments, a
cluster's center point corresponds to \emph{the average customer of that
segment}. Since the data is currently reduced in dimension and scaled by
a logarithm, we can recover the representative customer spending from
these data points by applying the inverse transformations.

In the code block below, you will need to implement the following: -
Apply the inverse transform to \texttt{centers} using
\texttt{pca.inverse\_transform} and assign the new centers to
\texttt{log\_centers}. - Apply the inverse function of \texttt{np.log}
to \texttt{log\_centers} using \texttt{np.exp} and assign the true
centers to \texttt{true\_centers}.

    \begin{Verbatim}[commandchars=\\\{\}]
{\color{incolor}In [{\color{incolor}18}]:} \PY{c+c1}{\PYZsh{} Inverse transform the centers}
         \PY{n}{log\PYZus{}centers} \PY{o}{=} \PY{n}{pca}\PY{o}{.}\PY{n}{inverse\PYZus{}transform}\PY{p}{(}\PY{n}{centers}\PY{p}{)}
         
         \PY{c+c1}{\PYZsh{} Exponentiate the centers}
         \PY{n}{true\PYZus{}centers} \PY{o}{=} \PY{n}{np}\PY{o}{.}\PY{n}{exp}\PY{p}{(}\PY{n}{log\PYZus{}centers}\PY{p}{)}
         
         \PY{c+c1}{\PYZsh{} Display the true centers}
         \PY{n}{segments} \PY{o}{=} \PY{p}{[}\PY{l+s+s1}{\PYZsq{}}\PY{l+s+s1}{Segment \PYZob{}\PYZcb{}}\PY{l+s+s1}{\PYZsq{}}\PY{o}{.}\PY{n}{format}\PY{p}{(}\PY{n}{i}\PY{p}{)} \PY{k}{for} \PY{n}{i} \PY{o+ow}{in} \PY{n+nb}{range}\PY{p}{(}\PY{l+m+mi}{0}\PY{p}{,}\PY{n+nb}{len}\PY{p}{(}\PY{n}{centers}\PY{p}{)}\PY{p}{)}\PY{p}{]}
         \PY{n}{true\PYZus{}centers} \PY{o}{=} \PY{n}{pd}\PY{o}{.}\PY{n}{DataFrame}\PY{p}{(}\PY{n}{np}\PY{o}{.}\PY{n}{round}\PY{p}{(}\PY{n}{true\PYZus{}centers}\PY{p}{)}\PY{p}{,} \PY{n}{columns} \PY{o}{=} \PY{n}{data}\PY{o}{.}\PY{n}{keys}\PY{p}{(}\PY{p}{)}\PY{p}{)}
         \PY{n}{true\PYZus{}centers}\PY{o}{.}\PY{n}{index} \PY{o}{=} \PY{n}{segments}
         \PY{n}{display}\PY{p}{(}\PY{n}{true\PYZus{}centers}\PY{p}{)}
         
         \PY{c+c1}{\PYZsh{} Plot bar graphs of each segment for a more visual interpretation}
         \PY{n}{pl}\PY{o}{.}\PY{n}{figure}\PY{p}{(}\PY{n}{figsize}\PY{o}{=}\PY{p}{(}\PY{l+m+mi}{10}\PY{p}{,}\PY{l+m+mi}{8}\PY{p}{)}\PY{p}{)}
         \PY{n}{pl}\PY{o}{.}\PY{n}{subplot}\PY{p}{(}\PY{l+m+mi}{211}\PY{p}{)}
         \PY{n}{pl}\PY{o}{.}\PY{n}{bar}\PY{p}{(}\PY{n+nb}{range}\PY{p}{(}\PY{n+nb}{len}\PY{p}{(}\PY{n}{true\PYZus{}centers}\PY{o}{.}\PY{n}{columns}\PY{p}{)}\PY{p}{)}\PY{p}{,} \PY{n}{true\PYZus{}centers}\PY{o}{.}\PY{n}{loc}\PY{p}{[}\PY{l+s+s1}{\PYZsq{}}\PY{l+s+s1}{Segment 0}\PY{l+s+s1}{\PYZsq{}}\PY{p}{]}\PY{p}{,} \PY{n}{tick\PYZus{}label}\PY{o}{=}\PY{n}{true\PYZus{}centers}\PY{o}{.}\PY{n}{columns}\PY{p}{,} \PY{n}{align}\PY{o}{=}\PY{l+s+s1}{\PYZsq{}}\PY{l+s+s1}{center}\PY{l+s+s1}{\PYZsq{}}\PY{p}{)}
         \PY{n}{pl}\PY{o}{.}\PY{n}{ylim}\PY{p}{(}\PY{p}{[}\PY{l+m+mi}{0}\PY{p}{,}\PY{l+m+mi}{14000}\PY{p}{]}\PY{p}{)}
         \PY{n}{pl}\PY{o}{.}\PY{n}{title}\PY{p}{(}\PY{l+s+s1}{\PYZsq{}}\PY{l+s+s1}{Segment 0}\PY{l+s+s1}{\PYZsq{}}\PY{p}{)}
         \PY{n}{pl}\PY{o}{.}\PY{n}{subplot}\PY{p}{(}\PY{l+m+mi}{212}\PY{p}{)}
         \PY{n}{pl}\PY{o}{.}\PY{n}{bar}\PY{p}{(}\PY{n+nb}{range}\PY{p}{(}\PY{n+nb}{len}\PY{p}{(}\PY{n}{true\PYZus{}centers}\PY{o}{.}\PY{n}{columns}\PY{p}{)}\PY{p}{)}\PY{p}{,} \PY{n}{true\PYZus{}centers}\PY{o}{.}\PY{n}{loc}\PY{p}{[}\PY{l+s+s1}{\PYZsq{}}\PY{l+s+s1}{Segment 1}\PY{l+s+s1}{\PYZsq{}}\PY{p}{]}\PY{p}{,} \PY{n}{tick\PYZus{}label}\PY{o}{=}\PY{n}{true\PYZus{}centers}\PY{o}{.}\PY{n}{columns}\PY{p}{,} \PY{n}{align}\PY{o}{=}\PY{l+s+s1}{\PYZsq{}}\PY{l+s+s1}{center}\PY{l+s+s1}{\PYZsq{}}\PY{p}{)}
         \PY{n}{pl}\PY{o}{.}\PY{n}{ylim}\PY{p}{(}\PY{p}{[}\PY{l+m+mi}{0}\PY{p}{,}\PY{l+m+mi}{14000}\PY{p}{]}\PY{p}{)}
         \PY{n}{pl}\PY{o}{.}\PY{n}{title}\PY{p}{(}\PY{l+s+s1}{\PYZsq{}}\PY{l+s+s1}{Segment 1}\PY{l+s+s1}{\PYZsq{}}\PY{p}{)}
         \PY{n}{pl}\PY{o}{.}\PY{n}{show}\PY{p}{(}\PY{p}{)}
\end{Verbatim}

    
    \begin{verbatim}
           Fresh  Milk  Grocery  Frozen  Detergents_Paper  Delicatessen
Segment 0   8867  1897     2477    2088               294           681
Segment 1   4005  7900    12104     952              4561          1036
    \end{verbatim}

    
    \begin{center}
    \adjustimage{max size={0.9\linewidth}{0.9\paperheight}}{customer_segments_files/customer_segments_53_1.png}
    \end{center}
    { \hspace*{\fill} \\}
    
    \subsubsection{Question 8}\label{question-8}

Consider the total purchase cost of each product category for the
representative data points above, and reference the statistical
description of the dataset at the beginning of this project. \emph{What
set of establishments could each of the customer segments represent?}\\
\textbf{Hint:} A customer who is assigned to
\texttt{\textquotesingle{}Cluster\ X\textquotesingle{}} should best
identify with the establishments represented by the feature set of
\texttt{\textquotesingle{}Segment\ X\textquotesingle{}}.

    \textbf{Answer:}

Judging based on the relative amounts of total purchase cost (from the
bar graphs above) for each product category, and referencing them
against the statistical description of the dataset: 1. Segment 0 has the
highest spending on `Fresh' and `Frozen', falling in the 3rd quartile of
the distribution of the entire dataset, followed by other product
categories in the 2nd quartile. Since relatively fresh and frozen
products are the most consumed, this segment can most likely be
associated with retail type of establishment. 2. Segment 1 has the
highest spending on `Milk', `Grocery' and `Detergents\_Paper', lying in
the 4th quartile of the dataset distribution, trailed by other
categories below their respective means. Thus, the higher usage of milk,
grocery and detergents products could be associated with a hotel.

    \subsubsection{Question 9}\label{question-9}

\emph{For each sample point, which customer segment from}
\textbf{\emph{Question 8}} \emph{best represents it? Are the predictions
for each sample point consistent with this?}

Run the code block below to find which cluster each sample point is
predicted to be.

    \begin{Verbatim}[commandchars=\\\{\}]
{\color{incolor}In [{\color{incolor}19}]:} \PY{c+c1}{\PYZsh{} Display the predictions}
         \PY{k}{for} \PY{n}{i}\PY{p}{,} \PY{n}{pred} \PY{o+ow}{in} \PY{n+nb}{enumerate}\PY{p}{(}\PY{n}{sample\PYZus{}preds}\PY{p}{)}\PY{p}{:}
             \PY{k}{print} \PY{l+s+s2}{\PYZdq{}}\PY{l+s+s2}{Sample point}\PY{l+s+s2}{\PYZdq{}}\PY{p}{,} \PY{n}{i}\PY{p}{,} \PY{l+s+s2}{\PYZdq{}}\PY{l+s+s2}{predicted to be in Cluster}\PY{l+s+s2}{\PYZdq{}}\PY{p}{,} \PY{n}{pred}
             
         \PY{c+c1}{\PYZsh{} Draw bar graphs of each sample customer to compare against the clustered segments}
         \PY{n}{pl}\PY{o}{.}\PY{n}{figure}\PY{p}{(}\PY{n}{figsize}\PY{o}{=}\PY{p}{(}\PY{l+m+mi}{10}\PY{p}{,}\PY{l+m+mi}{12}\PY{p}{)}\PY{p}{)}
         \PY{n}{pl}\PY{o}{.}\PY{n}{subplot}\PY{p}{(}\PY{l+m+mi}{311}\PY{p}{)}
         \PY{n}{pl}\PY{o}{.}\PY{n}{bar}\PY{p}{(}\PY{n+nb}{range}\PY{p}{(}\PY{n+nb}{len}\PY{p}{(}\PY{n}{samples}\PY{o}{.}\PY{n}{columns}\PY{p}{)}\PY{p}{)}\PY{p}{,} \PY{n}{samples}\PY{o}{.}\PY{n}{loc}\PY{p}{[}\PY{l+m+mi}{0}\PY{p}{]}\PY{p}{,} \PY{n}{tick\PYZus{}label}\PY{o}{=}\PY{n}{samples}\PY{o}{.}\PY{n}{columns}\PY{p}{,} \PY{n}{align}\PY{o}{=}\PY{l+s+s1}{\PYZsq{}}\PY{l+s+s1}{center}\PY{l+s+s1}{\PYZsq{}}\PY{p}{)}
         \PY{n}{pl}\PY{o}{.}\PY{n}{ylim}\PY{p}{(}\PY{p}{[}\PY{l+m+mi}{0}\PY{p}{,}\PY{l+m+mi}{14000}\PY{p}{]}\PY{p}{)}
         \PY{n}{pl}\PY{o}{.}\PY{n}{title}\PY{p}{(}\PY{l+s+s1}{\PYZsq{}}\PY{l+s+s1}{Sample 0}\PY{l+s+s1}{\PYZsq{}}\PY{p}{)}
         \PY{n}{pl}\PY{o}{.}\PY{n}{subplot}\PY{p}{(}\PY{l+m+mi}{312}\PY{p}{)}
         \PY{n}{pl}\PY{o}{.}\PY{n}{bar}\PY{p}{(}\PY{n+nb}{range}\PY{p}{(}\PY{n+nb}{len}\PY{p}{(}\PY{n}{samples}\PY{o}{.}\PY{n}{columns}\PY{p}{)}\PY{p}{)}\PY{p}{,} \PY{n}{samples}\PY{o}{.}\PY{n}{loc}\PY{p}{[}\PY{l+m+mi}{1}\PY{p}{]}\PY{p}{,} \PY{n}{tick\PYZus{}label}\PY{o}{=}\PY{n}{samples}\PY{o}{.}\PY{n}{columns}\PY{p}{,} \PY{n}{align}\PY{o}{=}\PY{l+s+s1}{\PYZsq{}}\PY{l+s+s1}{center}\PY{l+s+s1}{\PYZsq{}}\PY{p}{)}
         \PY{n}{pl}\PY{o}{.}\PY{n}{ylim}\PY{p}{(}\PY{p}{[}\PY{l+m+mi}{0}\PY{p}{,}\PY{l+m+mi}{14000}\PY{p}{]}\PY{p}{)}
         \PY{n}{pl}\PY{o}{.}\PY{n}{title}\PY{p}{(}\PY{l+s+s1}{\PYZsq{}}\PY{l+s+s1}{Sample 1}\PY{l+s+s1}{\PYZsq{}}\PY{p}{)}
         \PY{n}{pl}\PY{o}{.}\PY{n}{subplot}\PY{p}{(}\PY{l+m+mi}{313}\PY{p}{)}
         \PY{n}{pl}\PY{o}{.}\PY{n}{bar}\PY{p}{(}\PY{n+nb}{range}\PY{p}{(}\PY{n+nb}{len}\PY{p}{(}\PY{n}{samples}\PY{o}{.}\PY{n}{columns}\PY{p}{)}\PY{p}{)}\PY{p}{,} \PY{n}{samples}\PY{o}{.}\PY{n}{loc}\PY{p}{[}\PY{l+m+mi}{2}\PY{p}{]}\PY{p}{,} \PY{n}{tick\PYZus{}label}\PY{o}{=}\PY{n}{samples}\PY{o}{.}\PY{n}{columns}\PY{p}{,} \PY{n}{align}\PY{o}{=}\PY{l+s+s1}{\PYZsq{}}\PY{l+s+s1}{center}\PY{l+s+s1}{\PYZsq{}}\PY{p}{)}
         \PY{n}{pl}\PY{o}{.}\PY{n}{ylim}\PY{p}{(}\PY{p}{[}\PY{l+m+mi}{0}\PY{p}{,}\PY{l+m+mi}{14000}\PY{p}{]}\PY{p}{)}
         \PY{n}{pl}\PY{o}{.}\PY{n}{title}\PY{p}{(}\PY{l+s+s1}{\PYZsq{}}\PY{l+s+s1}{Sample 2}\PY{l+s+s1}{\PYZsq{}}\PY{p}{)}
         \PY{n}{pl}\PY{o}{.}\PY{n}{show}\PY{p}{(}\PY{p}{)}
\end{Verbatim}

    \begin{Verbatim}[commandchars=\\\{\}]
Sample point 0 predicted to be in Cluster 1
Sample point 1 predicted to be in Cluster 0
Sample point 2 predicted to be in Cluster 1
    \end{Verbatim}

    \begin{center}
    \adjustimage{max size={0.9\linewidth}{0.9\paperheight}}{customer_segments_files/customer_segments_57_1.png}
    \end{center}
    { \hspace*{\fill} \\}
    
    \textbf{Answer:}

The bar graphs of each product category for each sample is plotted
above. Comparing this against the bar graphs of the predicted segments,
it can be said that: 1. Sample point 0 matches very close to Segment 1
for `Milk', `Grocery' and `Detergents\_Paper' product categories.
Although `Fresh' differs by quite a bit in Sample 0, it shouldn't affect
overall classification significantly. 2. Sample point 1 has high
correlation with Segment 0, mainly due to `Fresh' and `Frozen' product
categories, although there is slight variance in `Grocery'. However, the
variance doesn't appear significant enough to affect the sample to be
categorized as Segment 0. 3. Sample point 2 has very high correlation
with Segment 1 for `Milk', `Grocery' and `Detergents\_Paper' product
categories, and the other categories have relatively insignificant
differences. Thus, this sample could clearly be classified as Segment 1.

These inferences are consistent with the the initial interpretations, as
well as the predictions of the clusterer.

    \paragraph{Additional:}\label{additional}

    \begin{Verbatim}[commandchars=\\\{\}]
{\color{incolor}In [{\color{incolor}20}]:} \PY{c+c1}{\PYZsh{} Make a combined heamap of samples and segments (cluster centers)}
         
         \PY{c+c1}{\PYZsh{} Append cluster centers to dataset}
         \PY{n}{new\PYZus{}data} \PY{o}{=} \PY{n}{data}\PY{o}{.}\PY{n}{append}\PY{p}{(}\PY{n}{true\PYZus{}centers}\PY{p}{)}
         
         \PY{c+c1}{\PYZsh{} Obtain percentiles of those centers}
         \PY{n}{ctr\PYZus{}pcts} \PY{o}{=} \PY{l+m+mf}{100.} \PY{o}{*} \PY{n}{new\PYZus{}data}\PY{o}{.}\PY{n}{rank}\PY{p}{(}\PY{n}{axis}\PY{o}{=}\PY{l+m+mi}{0}\PY{p}{,} \PY{n}{pct}\PY{o}{=}\PY{n+nb+bp}{True}\PY{p}{)}\PY{o}{.}\PY{n}{loc}\PY{p}{[}\PY{p}{[}\PY{l+s+s1}{\PYZsq{}}\PY{l+s+s1}{Segment 0}\PY{l+s+s1}{\PYZsq{}}\PY{p}{,} \PY{l+s+s1}{\PYZsq{}}\PY{l+s+s1}{Segment 1}\PY{l+s+s1}{\PYZsq{}}\PY{p}{]}\PY{p}{]}\PY{o}{.}\PY{n}{round}\PY{p}{(}\PY{n}{decimals}\PY{o}{=}\PY{l+m+mi}{3}\PY{p}{)}
         \PY{k}{print} \PY{l+s+s2}{\PYZdq{}}\PY{l+s+s2}{Centers:}\PY{l+s+s2}{\PYZdq{}}
         \PY{k}{print} \PY{n}{ctr\PYZus{}pcts}
         \PY{k}{print} \PY{l+s+s2}{\PYZdq{}}\PY{l+s+s2}{Samples:}\PY{l+s+s2}{\PYZdq{}}
         \PY{k}{print} \PY{n}{pcts}
         
         \PY{c+c1}{\PYZsh{} Visualize on a heatmap}
         \PY{n}{pl}\PY{o}{.}\PY{n}{figure}\PY{p}{(}\PY{n}{figsize}\PY{o}{=}\PY{p}{(}\PY{l+m+mi}{12}\PY{p}{,}\PY{l+m+mi}{8}\PY{p}{)}\PY{p}{)}
         \PY{n}{sns}\PY{o}{.}\PY{n}{heatmap}\PY{p}{(}\PY{n}{pcts}\PY{o}{.}\PY{n}{append}\PY{p}{(}\PY{n}{ctr\PYZus{}pcts}\PY{p}{)}\PY{p}{,} \PY{n}{annot}\PY{o}{=}\PY{n+nb+bp}{True}\PY{p}{,} \PY{n}{cmap}\PY{o}{=}\PY{l+s+s1}{\PYZsq{}}\PY{l+s+s1}{Greens}\PY{l+s+s1}{\PYZsq{}}\PY{p}{,} \PY{n}{fmt}\PY{o}{=}\PY{l+s+s1}{\PYZsq{}}\PY{l+s+s1}{0.1f}\PY{l+s+s1}{\PYZsq{}}\PY{p}{)}
         \PY{n}{pl}\PY{o}{.}\PY{n}{xticks}\PY{p}{(}\PY{n}{rotation}\PY{o}{=}\PY{l+m+mi}{45}\PY{p}{,} \PY{n}{ha}\PY{o}{=}\PY{l+s+s1}{\PYZsq{}}\PY{l+s+s1}{center}\PY{l+s+s1}{\PYZsq{}}\PY{p}{)}
         \PY{n}{pl}\PY{o}{.}\PY{n}{title}\PY{p}{(}\PY{l+s+s1}{\PYZsq{}}\PY{l+s+s1}{Category spending percentile ranks of}\PY{l+s+se}{\PYZbs{}n}\PY{l+s+s1}{Samples and Segment centers}\PY{l+s+s1}{\PYZsq{}}\PY{p}{)}
         \PY{n}{pl}\PY{o}{.}\PY{n}{show}\PY{p}{(}\PY{p}{)}
\end{Verbatim}

    \begin{Verbatim}[commandchars=\\\{\}]
Centers:
           Fresh  Milk  Grocery  Frozen  Detergents\_Paper  Delicatessen
Segment 0   52.3  30.5     30.5    58.7              28.3          37.6
Segment 1   28.7  79.0     79.9    35.1              79.2          52.9
Samples:
     Fresh  Milk  Grocery  Frozen  Detergents\_Paper  Delicatessen
100   62.5  78.6     80.2    73.0              91.4          90.5
198   63.0  34.8     60.9    74.8              43.4          38.9
305    3.6  91.6     69.5    27.0              75.0          11.6
    \end{Verbatim}

    \begin{center}
    \adjustimage{max size={0.9\linewidth}{0.9\paperheight}}{customer_segments_files/customer_segments_60_1.png}
    \end{center}
    { \hspace*{\fill} \\}
    
    \subsection{Conclusion}\label{conclusion}

    \subsubsection{Question 10}\label{question-10}

\emph{Companies often run
\href{https://en.wikipedia.org/wiki/A/B_testing}{A/B tests} when making
small changes to their products or services. If the wholesale
distributor wanted to change its delivery service from 5 days a week to
3 days a week, how would you use the structure of the data to help them
decide on a group of customers to test?}\\
\textbf{Hint:} Would such a change in the delivery service affect all
customers equally? How could the distributor identify who it affects the
most?

    \textbf{Answer:}

One of the possible ways of conducting an A/B test with the given
clustered dataset is to test the new delivery service with half of the
entire customer dataset, keeping the other half as control group. An
experimental group needs to be created by randomly sampling half of the
customers from both the clusters. (It is important to randomly sample to
avoid any biases.) The other half serves as the control group. The
experimental group then should be tested with the new delivery service
cycle, and the control group should be kept on the old delivery cycle.
Feedback from each group should then be taken about the customer
satisfaction with the ongoing delivery service. It is instrumental to
take feedback from the control group as well as the experimental group.
From the feedback, any potential resistance from a type of business (ex:
hotels that would prefer a steady supply of 5 days a week instead of 3
days a week) can be assessed from a particular cluster of the
experimental group that provides significant customer dissatisfaction.
More importantly, this result from experimental group should be
corroborated by minimal customer dissatisfaction with the control group
from the same cluster. Thus, the cluster that suffered from the change
in delivery service in the experimental group can be generalized to
apply to the entire cluster.

It is important to note that the controlled experiment mentioned above
could be rather simplistic, and in reality, nuances such as a particular
subset of a given cluster might need to be addressed or biased towards.
This may be eased by introducing more features to the dataset such as
total purchase cost to assess the overall size, or region (urban/rural)
etc.

    \subsubsection{Question 11}\label{question-11}

\emph{Assume the wholesale distributor wanted to predict some other
feature for each customer based on the purchasing information available.
How could the wholesale distributor use the structure of the data to
assist a supervised learning analysis?}

    \textbf{Answer:}

To aid further analysis, the cluster a customer belongs to (or customer
segment) could be used as an additional feature in a supervised learning
algorithm. A contrived example could be derived from the delivery
service cycle change example from above. Consider that ground truth from
a survey that furnishes ``Readiness to adapt to 3 day per week cycle''
for a few customers is obtained. This \emph{feature} can be used as a
\emph{label} to train a supervised learning algorithm, with an
additional input indicating the cluster a customer belongs to. If some
correlation exists, then the learning algorithm will be able to use the
information of customer segment to reach better accuracies in predicting
``Readiness to adapt to 3 day per week cycle'' for all customers.

    \subsubsection{Visualizing Underlying
Distributions}\label{visualizing-underlying-distributions}

At the beginning of this project, it was discussed that the
\texttt{\textquotesingle{}Channel\textquotesingle{}} and
\texttt{\textquotesingle{}Region\textquotesingle{}} features would be
excluded from the dataset so that the customer product categories were
emphasized in the analysis. By reintroducing the
\texttt{\textquotesingle{}Channel\textquotesingle{}} feature to the
dataset, an interesting structure emerges when considering the same PCA
dimensionality reduction applied earlier on to the original dataset.

Run the code block below to see how each data point is labeled either
\texttt{\textquotesingle{}HoReCa\textquotesingle{}}
(Hotel/Restaurant/Cafe) or
\texttt{\textquotesingle{}Retail\textquotesingle{}} the reduced space.
In addition, you will find the sample points are circled in the plot,
which will identify their labeling.

    \begin{Verbatim}[commandchars=\\\{\}]
{\color{incolor}In [{\color{incolor}21}]:} \PY{c+c1}{\PYZsh{} Display the clustering results based on \PYZsq{}Channel\PYZsq{} data}
         \PY{n}{rs}\PY{o}{.}\PY{n}{channel\PYZus{}results}\PY{p}{(}\PY{n}{reduced\PYZus{}data}\PY{p}{,} \PY{n}{outliers}\PY{p}{,} \PY{n}{pca\PYZus{}samples}\PY{p}{)}
\end{Verbatim}

    \begin{center}
    \adjustimage{max size={0.9\linewidth}{0.9\paperheight}}{customer_segments_files/customer_segments_67_0.png}
    \end{center}
    { \hspace*{\fill} \\}
    
    \subsubsection{Question 12}\label{question-12}

\emph{How well does the clustering algorithm and number of clusters
you've chosen compare to this underlying distribution of
Hotel/Restaurant/Cafe customers to Retailer customers? Are there
customer segments that would be classified as purely `Retailers' or
`Hotels/Restaurants/Cafes' by this distribution? Would you consider
these classifications as consistent with your previous definition of the
customer segments?}

    \textbf{Answer:}

The clustering algorithm realized above that separates the customers
into 2 groups, correlates well with the above visualization. There are a
few data points of the
\texttt{\textquotesingle{}Channel\textquotesingle{}} feature that are
interspersed in the regions clustering algorithm has predicted to be
homogeneous, but they don't seem to affect the \emph{bigger picture}.
Such data points however would be hard to categorize accurately with
most common clustering algorithms.

It is worthy to note that, although the objective identification of two
clusters has been performed, with the algorithm being completely
agnostic to \texttt{\textquotesingle{}Channel\textquotesingle{}}, there
is an inversion in the \textbf{subjective interpretation} of
\texttt{\textquotesingle{}Channel\textquotesingle{}}. That is, \emph{my}
subjective analysis of the dataset and its various tranformations such
as the initial interpretation, PCA component etc., is completely
opposite to the feature
\texttt{\textquotesingle{}Channel\textquotesingle{}}. My interpretation
of a `Hotel/Restaurant/Cafe' from the relative propotions of the total
purchase costs turns out to be `Retailers' according to
\texttt{\textquotesingle{}Channel\textquotesingle{}}. This is why, to my
interpretation, Sample 0 and 2 are hotels and Sample 1 is a retailer,
whereas, \texttt{\textquotesingle{}Channel\textquotesingle{}} indicates
that Sample 0 and 2 belongs to `Retailers' and Sample 1 belongs to
`Hotel/Restaurant/Cafe'. This doesn't adversely affect the integrity of
the project however as the \emph{name} of a customer segment is
irrelevant to the underlying data structure.


    % Add a bibliography block to the postdoc
    
    
    
    \end{document}
